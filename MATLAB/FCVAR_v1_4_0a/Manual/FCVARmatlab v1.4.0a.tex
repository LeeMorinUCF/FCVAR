\documentclass[10pt]{article}
\usepackage{amsmath}
\usepackage{amsfonts}
\usepackage{amssymb}
\usepackage{graphicx}
\usepackage{epstopdf}
\usepackage[margin=1in]{geometry}
\usepackage[compact,medium]{titlesec}%
\usepackage{natbib}             %for citations in text

\usepackage[bookmarks=true, citecolor=blue, colorlinks=true, urlcolor=blue]{hyperref}  % hyperlinks
\usepackage{mathrsfs} % for script fonts
\usepackage{float}   % used for better positioning of figures
\usepackage[FIGTOPCAP]{subfigure}     %FIGTOPCAP puts captions on top of subfigures
\usepackage{listings} % for syntax highlighting
\usepackage{color} %red, green, blue, yellow, cyan, magenta, black, white
%\definecolor{mygreen}{RGB}{28,172,0} % color values Red, Green, Blue
\definecolor{mygreen}{RGB}{0,102,0}
\definecolor{mylilas}{RGB}{170,55,241}

\lstset{language=Matlab,%
    %basicstyle=\color{red},
  % basicstyle=\footnotesize,   
  basicstyle={\ttfamily \small},   
    breaklines=true,%
    morekeywords={matlab2tikz},
    keywordstyle=\color{blue},%
    morekeywords=[2]{1}, keywordstyle=[2]{\color{black}},
    identifierstyle=\color{black},%
    stringstyle=\color{mylilas},
    commentstyle=\color{mygreen},%
    showstringspaces=false,%without this there will be a symbol in the places where there is a space
    numbers=left,  %
    % firstnumber={auto|last{number}},
    numberstyle={\tiny \color{black}},% size of the numbers
    numbersep=9pt, % this defines how far the numbers are from the text
    emph=[1]{for,end,break},emphstyle=[1]\color{red} %some words to emphasise
    %emph=[2]{word1,word2}, emphstyle=[2]{style},    
}

\def\lrt{{\rm LR}_T}


\begin{document}
 
\title{A Matlab program and user's guide for\\the fractionally cointegrated VAR model\thanks{We are grateful to Federico Carlini, Andreas Noack Jensen, S\o ren Johansen, Maggie Jones, James MacKinnon, Jason Rhinelander, and Daniela Osterrieder for comments, and to the Canada Research Chairs program, the Social Sciences and Humanities Research Council of Canada (SSHRC), and the Center for Research in Econometric Analysis of Time Series (CREATES, funded by the Danish National Research Foundation DNRF78) for financial support.}
\bigskip
\\version 1.4.0a
\bigskip} 
\author{Morten \O rregaard Nielsen\thanks{Corresponding author. If you find any bugs or other problems, please let us know.} \\
Queen's University and CREATES \\
Email:\ \texttt{mon@econ.queensu.ca}
\and Micha\l{} Ksawery Popiel \\
Queen's University \\
Email:\ \texttt{popielm@econ.queensu.ca}}


\maketitle

\begin{abstract}
This manual describes the usage of the accompanying freely available Matlab program for estimation and testing in the fractionally cointegrated vector autoregressive (FCVAR) model. This program replaces an earlier Matlab program by \cite{Nielsen2013}, and although the present Matlab program is not compatible with the earlier one, we encourage use of the new program.

\bigskip \noindent \textbf{JEL codes:} C22, C32.

\medskip \noindent \textbf{Keywords:} cofractional process, cointegration rank, computer program, fractional autoregressive model, fractional cointegration, fractional unit root, Matlab, VAR model.

\end{abstract}

\newpage

\tableofcontents

\newpage

\section{Obtaining and using the software\label{sec obtaining}}


\subsection{Disclaimer}

We have done our best to make this program as functional and free from errors as possible, but no warranty is given whatsoever. We cannot guarantee that we have been 100\% successful in eliminating bugs, so if you find any please let us know. 


\subsection{Obtaining the Matlab program}

The Matlab program can be downloaded from the first author's website at Queen's University:

\begin{center} \url{http://www.econ.queensu.ca/faculty/mon/software/}
\end{center}

\noindent It is freely available for non-commercial, academic use. For a (nearly) complete change log, please see Appendix \ref{sec version}.

Although the Matlab program can run standalone, one of the functions, \verb|RankTests.m|, makes an external system call to a separately installed program, \verb|fdpval|. This external program is the C++ implementation of a Fortran program used to obtain simulated $P$-values from \cite{mackinnon2014numerical}. If the user would like $P$-values for the cointegration rank tests to be automatically calculated, we recommend obtaining this companion program, which is made available by Jason Rhinelander and can be downloaded from: 
\begin{center} \url{https://github.com/jagerman/fracdist/releases}
\end{center}
It can be either installed or downloaded in a compressed folder. It is important to note where the program is stored or installed, because the Matlab program requires the program location as an input in the estimation options. For example, if the program is stored in the folder \verb|/usr/bin/| on a Linux system, the location variable is defined as follows, \verb|progLoc = '"/usr/bin/fdpval"'|. For details see Sections~\ref{sec:estoptions.m} and~\ref{sec:getpvals}.

We also make use of the excellent \verb|extrema.m| and \verb|extrema2.m| functions, which are written by Carlos Adrián Vargas Aguilera and are freely available from the Mathworks website. For simplicity these are included in the Auxiliary subfolder.

\subsection{Citation}

If you use this program, or any program or computer codes based on it, we ask that you please cite this document. For example, you could add ``The results were obtained using the computer program by Nielsen and Popiel (2016)" in the main text or in a footnote of your paper and then add the following citation to your list of references:\\

\noindent Nielsen, M. \O. and M. K. Popiel (2016),\ ``A Matlab program and user's guide for the fractionally cointegrated VAR model," QED working paper 1330, Queen's University.
\\

\subsection{Using the Matlab program}

The use of this program requires a functioning installation of Matlab. Any recent version should work. Unzip the contents of the zip file into any directory which will be the working directory of the program.

The next section describes the FCVAR model and the restricted models that can be estimated with this program. Section \ref{sec main} describes the functioning of the main program, which is a replication of one of the tables of results in \cite{JNP2014}. Section \ref{sec:examples} describes another example program, which demonstrates some additional functionality of the software. Importantly, these are the only two files that would need to be changed to apply the program for other empirical analyses. Section \ref{sec programs} describes how each of the major program files work (each in a separate subsection). The Appendix contains a version change log.


\newpage

\section{The fractionally cointegrated VAR model}
\label{sec fcvar}

The fractionally cointegrated vector autoregressive (FCVAR) model was proposed in \cite{Johansen2008} and analyzed by, e.g., \cite{johniel2010,johansen2012likelihood}. For a time series $X_{t}$ of dimension $p$, the fractionally cointegrated VAR model is given in error correction form as
\begin{equation}
\Delta^{d}X_{t}= \alpha \beta^{\prime} \Delta^{d-b} L_{b} X_{t} + 
\sum_{i=1}^{k}\Gamma_{i}\Delta^{d}\ L_{b}^{i}X_{t}
+ \varepsilon_{t},
\label{vecm model}%
\end{equation}
where $\varepsilon_{t}$ is $p$-dimensional $i.i.d.(0,\Omega)$, $\Delta^{d}$ is the fractional difference operator, and $L_{b}=1-\Delta^{b}$ is the fractional lag operator.\footnote{Both the fractional difference and fractional lag operators are defined in terms of their binomial expansion in the lag operator, $L$. Note that the expansion of $L_{b}$ has no term in $L^{0}$ and thus only lagged disequilibrium errors appear in \eqref{vecm model}.} \cite{johansen2012likelihood} imposed two restrictions on the parameter space, $d\geq b$ and $d-b<1/2$, in their asymptotic analysis. However, these restrictions were relaxed in \cite{JN2018b,JN2018}.

Model \eqref{vecm model} includes the \cite{Johansen1995} CVAR model as the special case $d=b=1$; see \cite{JN2018}. Some of the parameters are well-known from the CVAR model and these have the usual interpretations also in the FCVAR model. The most important of these are the long-run parameters $\alpha$ and $\beta$, which are $p \times r$ matrices with $0 \leq r \leq p$. The rank $r$ is termed the cointegration, or cofractional, rank. The columns of $\beta$ constitute the $r$ cointegration (cofractional) vectors such that $\beta' X_t$ are the cointegrating combinations of the variables in the system, i.e.\ the long-run equilibrium relations. The parameters in $\alpha$ are the adjustment or loading coefficients which represent the speed of adjustment towards equilibrium for each of the variables. The short-run dynamics of the variables are governed by the parameters $\Gamma=(\Gamma _{1},\ldots ,\Gamma _{k})$ in the autoregressive augmentation.

The FCVAR model has two additional parameters compared with the CVAR model, namely the fractional parameters $d$ and $b$. Here, $d$ denotes the fractional integration order of the observable time series and $b$ determines the degree of fractional cointegration, i.e.\ the reduction in fractional integration order of $\beta'X_t$ compared to $X_t$ itself. These parameters are estimated jointly with the remaining parameters. This model thus has the same main structure as in the standard CVAR model in that it allows for modeling of both cointegration and adjustment towards equilibrium, but is more general since it accommodates fractional integration and cointegration.

In the next four subsections we briefly describe the accommodation of deterministic terms as well as estimation and testing in the FCVAR model.


\subsection{Deterministic terms}

There are several ways to accommodate deterministic terms in the FCVAR model \eqref{vecm model}. The inclusion of the so-called restricted constant was considered in \cite{johansen2012likelihood}, and the so-called unrestricted constant term was considered in \cite{Dolatabadi2014}. A general formulation that encompasses both models is\footnote{In \cite{Dolatabadi2014} the constants are included as $\rho' \pi_t(1)$ and $\xi \pi_t(1)$, where $\pi_t(u)$ denotes coefficients in the binomial expansion of $(1-z)^{-u}$. This is mathematically convenient, but makes no difference in terms of the practical implementation.}
\begin{equation}
\Delta^{d}X_{t}= \alpha \Delta^{d-b} L_{b} (\beta^{\prime} X_{t} +\rho^{\prime}) + 
\sum_{i=1}^{k}\Gamma_{i}\Delta^{d}\ L_{b}^{i}X_{t} +\xi
+ \varepsilon_{t}.
\label{model with constants}
\end{equation}
The parameter $\rho $ is the so-called restricted constant term (since the constant term in the model is restricted to be of the form $\alpha \rho ^{\prime }$), which is interpreted as the mean level of the long-run equilibria when these are stationary, i.e.\ $E\beta ^{\prime }X_{t}+\rho ^{\prime }=0$. The parameter $\xi$ is the unrestricted constant term, which gives rise to a deterministic trend in the levels of the variables. When $d=1$ this trend is linear. Thus, the model \eqref{model with constants} contains both a restricted constant and an unrestricted constant. In the usual CVAR model, i.e.\ with $d=b=1$, the former would be absorbed in the latter, but in the fractional model they can both be present and are interpreted differently. For the representation theory related to \eqref{model with constants}, and in particular for additional interpretation of the two types of constant terms, see \cite{Dolatabadi2014}.

An alternative formulation of deterministic terms was suggested by \cite{johansen2014initial}, albeit in a simpler model, with the aim of reducing the impact of pre-sample observations of the process. This model is
\begin{equation}
\Delta^{d}(X_{t}-\mu)= \alpha \beta^{\prime} \Delta^{d-b} L_{b} (X_{t}-\mu) + 
\sum_{i=1}^{k}\Gamma_{i}\Delta^{d}\ L_{b}^{i}(X_{t}-\mu)
+ \varepsilon_{t},
\label{model with level par}
\end{equation}
which can be derived easily from the unobserved components formulation
\begin{equation}
X_t = \mu + X_t^0, \; \; \;
\Delta^{d}X_{t}^0=L_{b} \alpha \beta^{\prime} X_{t}^0 + 
\sum_{i=1}^{k}\Gamma_{i}\Delta^{d}\ L_{b}^{i}X_{t}^0
+ \varepsilon_{t}.
\label{UC model with level par}
\end{equation}
The formulation \eqref{model with level par}, or equivalently \eqref{UC model with level par}, includes the restricted constant, which may be obtained as $\rho'=\beta'\mu$. More generally, the level parameter $\mu$ is meant to accommodate a non-zero starting point for the first observation on the process, i.e., for $X_1$. It has the added advantage of reducing the bias arising due to pre-sample behavior of $X_t$, at least in simple models, even when conditioning on no initial values (see below). For details, see \cite{johansen2014initial}.


\subsection{Maximum likelihood estimation}

It is assumed that a sample of length $T+N$ is available on $X_t$, where $N$ denotes the number of observations used for conditioning, for details see \cite{johansen2014initial}. The models \eqref{vecm model}, \eqref{model with constants}, and \eqref{model with level par} are estimated by conditional maximum likelihood, conditional on $N$ initial values, by maximizing the function
\begin{equation}
\log L_{T}(\lambda ) = - \frac{T p}{2} ( \log(2\pi) + 1) 
-\frac{T}{2}\log \det \left\{ T^{-1}\sum_{t=N+1}^{T+N}\varepsilon _{t}(\lambda )\varepsilon _{t}(\lambda)^{\prime }\right\},  \label{Cond Like}
\end{equation}
where the residuals are defined as
\begin{equation}
\varepsilon _{t}(\lambda ) = \Delta^{d}X_{t}- \alpha \Delta^{d-b} L_{b} (\beta^{\prime} X_{t} + \rho') 
 - \sum_{i=1}^{k}\Gamma_{i}\Delta^{d}\ L_{b}^{i}X_{t} - \xi, \quad
\lambda =(d,b,\alpha,\beta,\Gamma,\rho,\xi),
\label{epstau}
\end{equation}
for model \eqref{model with constants}, and hence also for submodels of model \eqref{model with constants}, such as \eqref{vecm model}, with the appropriate restrictions imposed on $\rho$ and $\xi$. For model \eqref{model with level par} the residuals are
\begin{equation}
\varepsilon _{t}(\lambda ) = \Delta^{d}(X_{t}-\mu) -  \alpha \beta^{\prime} \Delta^{d-b} L_{b} (X_{t} - \mu) 
 - \sum_{i=1}^{k}\Gamma_{i}\Delta^{d}\ L_{b}^{i}(X_{t} - \mu), \quad
\lambda =(d,b,\alpha,\beta,\Gamma,\mu).
\label{epstau level}
\end{equation}
It is shown in \cite{johansen2012likelihood} and \cite{Dolatabadi2014} how, for fixed $(d,b)$, the estimation of model \eqref{model with constants} reduces to regression and reduced rank regression as in \cite{Johansen1995}. In this way the parameters $(\alpha ,\beta,\Gamma,\rho,\xi)$ can be concentrated out of the likelihood function, and numerical optimization is only needed to optimize the profile likelihood function over the two fractional parameters, $d$ and $b$. In model \eqref{model with level par} we can similarly concentrate the parameters $(\alpha ,\beta,\Gamma)$ out of the likelihood function resulting in numerical optimization over $(d,b,\mu)$, making the estimation of model \eqref{model with level par} slightly more involved numerically than that of model \eqref{model with constants}.

For model \eqref{model with constants} with $\xi=0$, \cite{johansen2012likelihood} shows that asymptotic theory is standard when $b<0.5$, and for the case $b>0.5$ asymptotic theory is non-standard and involves fractional Brownian motion of type II. Specifically, when $b>0.5$, \cite{johansen2012likelihood} shows that under i.i.d.\ errors with suitable moment conditions, the conditional maximum likelihood parameter estimates  $(\hat{d},\hat{b},\hat{\alpha},\hat{\Gamma}_{1},\ldots,\hat{\Gamma}_{k})$ are asymptotically Gaussian, while $(\hat{\beta},\hat{\rho})$\ are locally asymptotically mixed normal. These results allow asymptotically standard (chi-squared) inference on all parameters of the model, including the cointegrating relations and orders of fractionality, using quasi-likelihood ratio tests. As in the CVAR model, see \cite{Johansen1995}, the same results hold for the same parameters in the full models \eqref{model with constants} and \eqref{model with level par}, whereas the asymptotic distribution theory for the remaining parameters, $\xi$ and $\mu$, is currently unknown.


\subsection{Cointegration rank tests}

Letting $\Pi = \alpha \beta'$, the likelihood ratio (LR) test statistic of the hypothesis $\mathcal{H}_{r}: \mathrm{rank}(\Pi )=r$ against $\mathcal{H}_{p}:\mathrm{rank}(\Pi )=p$ is of particular interest because it deals with an important empirical question. This statistic is often denoted the ``trace'' statistic. Let $\theta = (d,b)$ for model \eqref{model with constants} and $\theta = (d,b,\mu)$ for model \eqref{model with level par} denote the parameters for which the likelihood is numerically maximized. Then let $L(\theta,r)$ be the profile likelihood function given rank $r$, where $(\alpha ,\beta ,\Gamma )$, and possibly $(\rho,\xi)$ if appropriate, have been concentrated out by regression and reduced rank regression; see \cite{johansen2012likelihood} and \cite{Dolatabadi2014} for details.

The profile likelihood function is maximized both under the hypothesis $\mathcal{H}_{r}$ and under $\mathcal{H}_{p}$ and the LR test statistic is then $\lrt(q)=2\log(L(\hat{\theta}_{p},p)/L(\hat{\theta}_{r},r))$, where 
\begin{equation*}
L(\hat{\theta}_{p},p)=\max_{\theta}L(\theta,p), \quad L(\hat{\theta}_{r},r)=\max_{\theta}L(\theta,r),
\end{equation*}
and $q=p-r$. This problem is qualitatively different from that in \cite{Johansen1995} since the asymptotic distribution of $\lrt(q)$ depends qualitatively (and quantitatively) on the parameter $b$. In the case with $0<b<1/2$ (sometimes known as \textquotedblleft weak cointegration\textquotedblright ), $\lrt(q)$ has a standard asymptotic distribution, see \citet[Theorem 11(ii)]{johansen2012likelihood}, namely 
\begin{equation}
\lrt(q)\overset{D}{\rightarrow }\chi ^{2}(q^{2}), \; \; 0<b<1/2.
\end{equation}
On the other hand, when $1/2<b\leq d$ (\textquotedblleft strong cointegration\textquotedblright ), asymptotic theory is nonstandard and 
\begin{equation}
\lrt(q)\overset{D}{\rightarrow }\mathrm{Tr}\left\{
\int_{0}^{1}\mathsf{d}W(s)F(s)^{\prime }\left( \int_{0}^{1}F(s)F(s)^{\prime}\mathsf{d}s\right) ^{-1}
\int_{0}^{1}F(s)\mathsf{d}W(s)^{\prime }\right\}, \; \; b>1/2,
\label{LR asy distr}
\end{equation}
where the vector process $\mathsf{d}W$ is the increment of ordinary (non-fractional) vector standard Brownian motion of dimension $q=p-r$. The vector process $F$ depends on the deterministics in a similar way as in the CVAR model in \cite{Johansen1995}, although the fractional orders complicate matters. The following cases have been derived in the literature:

 \begin{enumerate}
 \item When no deterministic term is in the model, $F(u)=W_{b}(u)$, where $W_{b}(u)=\Gamma(b)^{-1}\int_{0}^{u}(u-s)^{b-1}\mathsf{d}W(s)$ is vector fractional Brownian motion of type II, see \citet[Theorem 11(i)]{johansen2012likelihood}.
 \item When only the restricted constant term is included in model \eqref{model with constants}, $F(u)=(W_{b}(u)^{\prime},u^{-(d-b)})^{\prime }$, see \citet[Theorem 11(iv)]{johansen2012likelihood} for the result with $d=b$ and an earlier working paper version for the general result.
\item In model \eqref{model with level par} the same result as in bullet 2.\ holds because $\beta'\mu = \rho'$ is the restricted constant and $\beta_\perp^{\prime} \mu$ has no influence on the asymptotic distribution (in a similar way to $X_0$ in a random walk).
 \item When both the restricted and unrestricted constants are included in model \eqref{model with constants} with $d=1$,
\begin{align*}
F_{i}(u) &= W_{b,i}(u) - \int_{0}^{1} W_{b,i}(u) \mathsf{d} u, \ i=1,...,q-1, \\
F_{q}(u) &= u^{b} - \int_{0}^{1}u^{b} \mathsf{d} u=u^{b}-1/(b+1), \\
F_{q+1}(u) &= u^{b-1} - \int_{0}^{1}u^{b-1} \mathsf{d} u = u^{b-1}-1/b, 
\end{align*}
see \cite{Dolatabadi2014}.
\end{enumerate}

Importantly, the asymptotic distribution \eqref{LR asy distr} of the test statistic $\lrt(q)$ depends on both $b$ and $q=p-r$. The dependence on the unknown (true value of the) scalar parameter $b$ complicates empirical analysis compared to the CVAR model. Generally, the distribution \eqref{LR asy distr} would need to be simulated on a case-by-case basis. However, for model \eqref{vecm model} and for model \eqref{model with constants} with $d=b$ and $\xi=0$, and hence also for model \eqref{model with level par} with $d=b$ in light of bullet 3.\ above, computer programs for computing asymptotic critical values and asymptotic $P$ values for the LR cointegration rank tests based on numerical distribution functions, are made available by \cite{mackinnon2014numerical}. Their computer programs are incorporated in the present program for the relevant cases/models as discussed and illustrated below.


\subsection{Restricted models}
\label{restricted models}

Note that a reduced rank restriction has already been imposed on models \eqref{vecm model}--\eqref{model with level par}, where the coefficient matrix $\Pi  = \alpha \beta '$ has been restricted to rank $r \leq p$. Other restrictions on the model parameters can be considered as in \cite{Johansen1995}. The most interesting restrictions from an economic theory point of view would likely be restrictions on the adjustment parameters $\alpha$ and cointegration vectors $\beta$.

We formulate hypotheses as
\begin{align}
  &R_{\psi} \psi = r_{\psi}, \label{R psi} \\
  &R_\alpha \mathrm{vec}(\alpha) = 0, \label{R alpha} \\
  &R_\beta \mathrm{vec}(\beta^{\ast}) = r_\beta, \label{R beta}
\end{align}
with $\beta^{\ast} = (\beta', \rho')'$, and use the switching algorithm in \cite[p.\ 455]{Boswijk2004} to optimize the likelihood numerically subject to the restrictions. The switching algorithm can be improved by adding a line search, see \cite{Doornik2016}. This is done by setting the option \verb|opt.LineSearch = 1|, which is the default setting.

The only limitation on the linear restrictions that can be imposed on $(d,b,\alpha,\beta^{\ast})$ in \eqref{R psi}--\eqref{R beta} is that only homogenous restrictions can be imposed on $\mathrm{vec}(\alpha)$ in \eqref{R alpha}. Otherwise, any combination of linear restrictions can be imposed on these parameters. For now, the remaining parameters cannot be restricted.

Note that, when the restricted constant term $\rho$ is included in the model, restrictions on $\beta$ and $\rho$ must be written in the form given by \eqref{R beta}. This is without loss of generality.

The restrictions in \eqref{R psi}--\eqref{R beta} above can be implemented individually or simultaneously in the Matlab program. The next section provides an example session illustrating the use of the program with a step-by-step description of a typical empirical analysis, including several restricted models in Section \ref{subsec hypo}.


\subsection{Forecasting from the FCVAR model}

Because the FCVAR model is autoregressive, the best linear predictor takes a simple form and is relatively straightforward to calculate. Consider, for example, the model with level parameter in \eqref{model with level par}. We first note that
\begin{equation}
\Delta^d (X_{t+1}-\mu) = X_{t+1}-\mu  - (X_{t+1}-\mu) + \Delta^d (X_{t+1}-\mu) = X_{t+1}-\mu - L_d (X_{t+1}-\mu)
\nonumber
\end{equation}
and then rearrange (\ref{model with level par}) as
\begin{equation}
X_{t+1}=\mu + L_d (X_{t+1}-\mu)+\alpha\beta^{\prime}\Delta^{d-b}L_{b}(X_{t+1}-\mu)+\sum
_{i=1}^{k}\Gamma_{i}\Delta^d L_{b}^{i}(X_{t+1}-\mu)+\varepsilon_{t+1}.
\label{model forecast}
\end{equation}
Since $L_{b}=1-\Delta^{b}$ is a lag operator, so that $L_{b}^{i}X_{t+1}$ is known at time $t$ for $i \geq 1$, this equation can be used as the basis to calculate forecasts from the model.

We let conditional expectation given the information set at time $t$ be denoted $E_{t}(\cdot)$, and the best linear predictor forecast of any variable $Z_{t+1}$ given information available at time $t$ be denoted $\hat{Z}_{t+1|t}=E_{t}(Z_{t+1})$. Clearly, we then have that the forecast of the innovation for period $t+1$ at time $t$ is $\hat{\varepsilon}_{t+1|t}=E_{t}(\varepsilon_{t+1})=0$, and $\hat{X}_{t+1|t}$ is then easily found from (\ref{model forecast}). Inserting also coefficient estimates based on data available up to time $t$, denoted\footnote{To emphasize that these estimates are based on data available at time $t$, they could be denoted by a subscript $t$. However, to avoid cluttering the notation we omit this subscript and let it be understood in the sequel.} $(\hat{d},\hat{b},\hat{\mu},\hat{\alpha},\hat{\beta },\hat{\Gamma}_{1},\ldots,\hat{\Gamma}_{k})$, we have that
\begin{equation}
\hat{X}_{t+1|t}=\hat{\mu}+L_{\hat{d}} (X_{t+1}-\hat{\mu})+\hat{\alpha}\hat{\beta}^{\prime}\Delta^{\hat{d}-\hat{b}
}L_{\hat{b}}(X_{t+1}-\hat{\mu})+\sum_{i=1}^{k}\hat{\Gamma}_{i}\Delta^{\hat{d}}
L_{\hat{b}}^{i}(X_{t+1}-\hat{\mu}).
\label{forecast one}
\end{equation}
This defines the one-step ahead forecast of $X_{t+1}$ given information at time $t$. 

Multi-period ahead forecasts can be generated recursively. That is, to calculate the $h$-step ahead forecast, we first generalize (\ref{forecast one}) as
\begin{equation}
\hat{X}_{t+j|t}=\hat{\mu}+L_{\hat{d}} (\hat{X}_{t+j|t}-\hat{\mu})+\hat{\alpha}\hat{\beta}^{\prime}
\Delta^{\hat{d}-\hat{b}}L_{\hat{b}}(\hat{X}_{t+j|t}-\hat{\mu})+\sum_{i=1}^{k}
\hat{\Gamma}_{i}\Delta^{\hat{d}} L_{\hat{b}}^{i}(\hat{X}_{t+j|t}-\hat{\mu}),
\label{forecast multi}
\end{equation}
where $\hat{X}_{s|t}=X_{s}$ for $s\leq t$. Then forecasts are calculated recursively from (\ref{forecast multi}) for $j=1,2,\ldots,h$ to generate $h$-step ahead forecasts, $\hat{X}_{t+h|t}$.

Clearly, one-step ahead and $h$-step ahead forecasts for the model \eqref{model with constants} with a restricted constant term, and possibly also an unrestricted constant term, instead of the level parameter can be calculated entirely analogously.


\newpage

\section{Example session: replication\textunderscore{JNP2014}.m}
\label{sec main}


The main file is \verb|replication_JNP2014.m| and it serves as an example of what a typical session of estimation, testing, and forecasting can include. This code replicates ``Table 4: FCVAR results for Model 1'' from \cite{JNP2014} and follows the empirical procedure developed in that paper. This procedure includes the following steps:

\begin{enumerate}
\item Importing data
\item Choosing estimation options
\item Lag selection
\item Cointegration rank selection
\item Model estimation
\item Hypothesis testing
\end{enumerate}

It is important to note that all necessary commands for file execution and option modification can be called from this script. All other files contained in the package (described in detail in the next section) do not require any modification by the user.

To accommodate the sequential nature of the procedure, the main file is broken up into \textit{code sections}\footnote{For more information see \url{ http://www.mathworks.com/help/matlab/matlab_prog/run-sections-of-programs.html}}. These \textit{code sections}, known as \textit{cells} in previous versions of Matlab, allow the user to execute specific parts of a script individually. Each of the \textit{code sections} are delimited by a double comment \verb|%%| and the section header. 


\subsection{Importing data}

The first step is importing the data. Executing the code in Listing~\ref{data}, shown below, assigns the data from the file \verb|data_JNP2014.csv| to a matrix called \verb|data|. 

\lstset{firstnumber=last}
\begin{lstlisting}[frame=single,caption={Importing data}, label = data]
% -------- Import Data ----------%
clear all;
data = csvread('data_JNP2014.csv',1); % skip first row because var names.

% data for each model.
x1 = data(:, [1 3 5]);
x2 = data(:, [2 3 5]);
x3 = data(:, [1 2 3 5]);
x4 = data(:, [1 3 4 5 6]);
x5 = data(:, [2 3 4 5 6]);
x6 = data(:, [1 2 3 4 5 6]);
\end{lstlisting}

The columns contain the following variables: (1) aggregate support for the Liberal party, (2) aggregate support for the Conservative party, (3) Canadian 3-month T-bill rates, (4) US 3-month T-bill rates, (5) Canadian unemployment rate, and (6) US unemployment rate. Since each of the models in JNP (2014) contain different combinations of these variables, the relevant columns of \verb|data| for each model are assigned to different matrices of variables named \verb|x1| through \verb|x6|.

\subsection{Choosing options}
\label{subsec choosing options}

Once the data is imported, the user sets the program options. The script contains two sets of options: variables set for function arguments in the script itself and model/estimation related options. Listing ~\ref{intl} shows the first of set of options. 
\begin{lstlisting}[frame=single,caption={Initialization of local variables}, label = intl]
%% -------- INITIALIZATION ----------%
p                = size(x1, 2); % system dimension.
kmax             = 3;    % maximum number of lags for VECM.
order            = 12;   % number of lags for white noise test in lag selection.
printWNtest      = 1;    % to print results of  white noise tests post-estimation.
\end{lstlisting}

The variable \verb|kmax| determines the highest lag order for the sequential testing that is performed in the lag selection, whereas \verb|p| is the dimension of the system. The other variables are self-explanatory.

The next set of initialization commands, shown in Listing~\ref{intl2}, assign values to the variables contained in object \verb|opt| defined by the class \verb|EstOptions|. 

\begin{lstlisting}[frame=single,caption={Choosing estimation options}, label = intl2]
% -------- Choosing estimation options ----------%
opt = EstOptions; % Define variable to store Estimation Options (object).
opt.dbMin        = [0.01 0.01]; % lower bound for d,b.
opt.dbMax        = [2.00 2.00]; % upper bound for d,b.
opt.unrConstant  = 0; % include an unrestricted constant? 1 = yes, 0 = no.
opt.rConstant    = 0; % include a restricted constant? 1 = yes, 0 = no.
opt.levelParam   = 1; % include level parameter? 1 = yes, 0 = no.
opt.constrained  = 0; % impose restriction dbMax >= d >= b >= dbMin ? 
                      % 1 = yes, 0 = no.
opt.restrictDB   = 1; % impose restriction d=b ? 1 = yes, 0 = no.
opt.db0          = [.8 .8]; % set starting values for optimization algorithm.
opt.N            = 0; % number of initial values to condition upon.
opt.print2screen = 1; % print output.
opt.printRoots   = 1; % print roots of characteristic polynomial.
opt.plotRoots    = 1; % plot roots of characteristic polynomial.
opt.gridSearch   = 1; % For more accurate estimation, perform the grid search.
					  % This will make estimation take longer.
opt.plotLike     = 0; % Plot the likelihood (if gridSearch = 1).
opt.progress 	 = 0; % Show grid search progress indicator waitbar.
opt.updateTime   = 5; % How often progress is updated (seconds).

% Linux example:
opt.progLoc = '"/usr/bin/fdpval"';  % location path with program name 
                                    % of fracdist program, if installed
                                    % Note: use both single (outside) and double 
                                    % quotes (inside). This is especially important 
                                    % if path name has spaces.
DefaultOpt = opt; % Store the options for restoring them in between hypothesis tests.
\end{lstlisting}

The first line initializes the object \verb|opt| and assigns all of the default options set in \verb|EstOptions|. The user can see the full set of options by typing \verb|EstOptions| (or \verb|opt| after initialization) in the command line. Listing~\ref{intl2} shows how to easily change any of the default options. Defining the program options in this way allows the user to create and store several option objects with different attributes. This can be very convenient when, for example, performing the same hypothesis tests on different data sets. 

The set of available options can be broken into several categories: numerical optimization, model deterministics and restrictions, output, grid search, and $P$-values for the rank test. We recommend that only advanced users make changes to the numerical optimization options. Adding deterministics requires setting the variable corresponding to the type of deterministic component to 1. For instance, in the present example, a model estimated with options \verb|opt| will include the level parameter $\mu$ but no restricted or unrestricted constant. Output variables refer to either printing or plotting various information post-estimation and usually take values $1$ or $0$ (on or off). For example, if the user is not interested in the estimates of $\Gamma$, they can be suppressed by setting \verb|opt.printGammas = 0|.

The bounds on the parameter space for $d$ and $b$ are specified in \verb|opt.dbMin| and \verb|opt.dbMax|. In this example, these are both specified as 2-dimensional column vectors, in which case the first element specifies the bound on $d$ and the second element the bound on $b$. Alternatively, one can set \verb|opt.dbMin| and \verb|opt.dbMax| as scalars, which imposes the same bounds on $d$ and $b$. 

An important feature in this package is the ability to pre-estimate by using a grid search. If the user selects this option, they can view progress by setting \verb|opt.progress| to $1$ (waitbar) or $2$ (output in command line). The minimum frequency of these updates is set by \verb|opt.updateTime|. The user also has the option (\verb|opt.plotLike|) to view a plot of the likelihood over $d$ and/or $b$ after the grid search completes. The output of the grid search is a preliminary estimate of the fractional parameters. These are used as starting values in the subsequent numerical optimization, and the bounds on $d$ and $b$ are set to these starting values plus/minus $0.1$ but still within the original \verb|dbMin| and \verb|dbMax| settings.

As of v.1.4.0, the new option \verb|opt.LocalMax| allows more control over the grid search. If \verb|opt.LocalMax = 0|, the function \verb|LikeGrid.m| returns the parameter values corresponding to the global maximum of the likelihood on the grid. If \verb|opt.LocalMax = 1|, then \verb|LikeGrid.m| returns the parameter values for the local maximum corresponding to the highest value of $b$. This is meant to alleviate the identification problem discussed in \citet[Section 2.3]{johniel2010} and \cite{Carlini2014}. As of v.1.4.0, the default setting is \verb|opt.LocalMax = 1|.

Another new option in v.1.4.0 is the addition of a line search to the switching algorithm for estimation of models with restrictions on $\alpha$ and/or $\beta$. This is added via the option \verb|opt.LineSearch = 1| and is the default. See \citet[Section 2.2]{Doornik2016} for details.

In order to automatically obtain $P$-values for cointegration rank tests when $b>0.5$, the user needs to download and install the necessary program (see Section~\ref{sec obtaining}). The last option, \verb|opt.progLoc|, identifies the location of that program.

After all options have been set, the last line in Listing~\ref{intl2} stores them in \verb|DefaultOpt| so that the user can recall them at any point in the estimation. This is particularly useful if the user wants to change only a few options in between estimations.


\subsection{Lag-order selection}

Once the options are set, the user moves to the next step, which involves choosing the appropriate lag order. The relevant information is obtained with a call to \verb|LagSelect.m|, shown in Listing~\ref{LagSelect}, which performs estimation of models with lag-orders from $0$ to \verb|kmax|. The program performs lag selection on the full-rank unrestricted model. 

\begin{lstlisting}[frame=single,caption={Lag selection}, label = LagSelect]
%% --------- LAG SELECTION ---------- %
LagSelect(x1, kmax, p, order, opt);
\end{lstlisting}

The output generated by this function is shown below. 

\begin{verbatim}
-----------------------------------------------------------------------------------------------------
                        Lag Selection Results 
-----------------------------------------------------------------------------------------------------
Dimension of system:       3     Number of observations in sample:          316 
Order for WN tests:       12     Number of observations for estimation:     316 
Restricted constant:      No     Initial values:                              0
Unrestricted constant:     No     Level parameter:                           Yes
-----------------------------------------------------------------------------------------------------
k  r    d    b      LogL     LR    pv    AIC       BIC     pmvQ pQ1  pLM1 pQ2  pLM2 pQ3  pLM3
 3  3 0.676 0.676  456.42   7.31 0.605  -832.85   -682.62  0.94 0.72 0.46 0.49 0.89 0.51 0.47
 2  3 0.581 0.581  452.77  20.59 0.015  -843.53*  -727.11  0.82 0.69 0.45 0.29 0.75 0.54 0.40
 1  3 1.043 1.043  442.47  56.99 0.000  -840.94   -758.31* 0.34 0.75 0.52 0.15 0.58 0.34 0.18
 0  3 1.036 1.036  413.97   0.00 0.000  -801.95   -753.12  0.00 0.01 0.01 0.00 0.08 0.37 0.17
-----------------------------------------------------------------------------------------------------
\end{verbatim}

Estimates of $d$ and $b$ are reported for each lag ($k$) with rank ($r$) set to the number of variables in the system. Note that in this example the restriction $d=b$ has been imposed. The log-likelihood for each lag is shown in column \verb|LogL|. The likelihood ratio test-statistic \verb|LR| is for the null hypothesis $\Gamma_k = 0$ with $P$-value reported in column \verb|pv|. This is followed by AIC and BIC information criteria. The next set of columns provides $P$-values for white noise tests on the residuals. The first $P$-value, \verb|pmvQ|, is for the multivariate Q-test followed by univarite Q-tests as well as LM tests on the $p$ individual residuals; that is, \verb|pQ1| and \verb|pLM1| are the $P$-values for the residuals in the first equation, \verb|pQ2| and \verb|pLM2| are for the residuals in the second equation, and so on. 


\subsection{Cointegration rank testing}

The user now chooses the lag-order based on the information provided above and can move to the next step, which is cointegration rank testing. The next code section is shown in listing~\ref{RankSelect}. The user first assigns the lag augmentation, $k=2$ in this case, and then calls the function \verb|RankTests.m|. 

\begin{lstlisting}[frame=single,caption={Cointegration rank testing}, label = RankSelect]
%% --------- COINTEGRATION RANK TESTING ---------- %
k = 2;
rankTestStats = RankTests(x1, k, opt);
\end{lstlisting}

Executing the code in Listing~\ref{RankSelect} produces the following output. 

\begin{verbatim}
-----------------------------------------------------------------------------------------------------
                         Likelihood Ratio Tests for Cointegrating Rank                               
-----------------------------------------------------------------------------------------------------
Dimension of system:       3     Number of observations in sample:          316 
Number of lags:            2     Number of observations for estimation:     316 
Restricted constant:      No     Initial values:                              0
Unestricted constant:     No     Level parameter:                           Yes
-----------------------------------------------------------------------------------------------------
Rank 	  d  	  b  	 Log-likelihood	 LR statistic	 P-value
 0   	0.643	0.643	        440.040	       25.454	   0.043
 1   	0.569	0.569	        451.174	        3.186	   0.820
 2   	0.576	0.576	        452.707	        0.120	   0.947
 3   	0.581	0.581	        452.767	         ----	    ----
-----------------------------------------------------------------------------------------------------
\end{verbatim}

The first block of output provides a summary of the model specification. The second block provides the test results relevant for selecting the appropriate rank. The table is meant to be read sequentially from lowest to highest rank, i.e.\ from top to bottom. Since we can reject the null of rank 0 against the alternative of rank 3 we move to the test of rank 1 against rank 3. This test fails to reject with a $P$-value of $0.820$, so this is the appropriate choice in this case.


\subsection{Unrestricted model estimation}

With the rank and lag selected, the user can now move to the next code section, shown in Listing~\ref{unrEst}.

\begin{lstlisting}[frame=single,caption={Unrestricted model estimation}, label = unrEst]
%% --------- UNRESTRICTED MODEL ESTIMATION ---------- %
r=1;

opt1 = DefaultOpt;

m1 = FCVARestn(x1, k, r, opt1); % This model is now in the structure m1.

mv_wntest(m1.Residuals, order, printWNtest);
\end{lstlisting}

Here the user first specifies the choice for the rank based on the previously performed cointegrating rank tests (thus setting $r=1$ in this example). Next, the default options set in the initialization, see Section \ref{subsec choosing options}, are assigned to \verb|opt1|, which is used as an argument in the call to the function \verb|FCVARestn.m|. This function is the main part of the program since it performs the estimation of the parameters, obtains model residuals and standard errors, and calculates many other relevant components such as the number of free parameters and the roots of the characteristic polynomial. If \verb|opt1.print2screen=1| then, in addition to storing all of these results in the Matlab structure \verb|m1|, the function outputs the estimation results to the command window. To see a list of variables stored in \verb|m1|, the user can type \verb|m1| in the command line. After the unrestricted model has been estimated, this code section concludes with a call to \verb|mv_wntest.m|, which performs a series of white noise tests on the residuals and prints the output in the command window.

The program output is shown below. It begins with a table summarizing relevant model specifications and then the coefficients and their standard errors. The roots of the characteristic polynomial are displayed at the bottom. 

\begin{verbatim}
-----------------------------------------------------------------------------------------------------
                      Fractionally Cointegrated VAR: Estimation Results                              
-----------------------------------------------------------------------------------------------------
Dimension of system:       3      Number of observations in sample:          316 
Number of lags:            2      Number of observations for estimation:     316 
Restricted constant:      No      Initial values:                              0
Unrestricted constant:    No      Level parameter:                           Yes
Starting value for d:    0.800    Parameter space for d: (0.010 , 2.000) 
Starting value for b:    0.800    Parameter space for b: (0.010 , 2.000) 
-----------------------------------------------------------------------------------------------------
Cointegrating rank:            1  AIC:              -848.348 
Log-likelihood:          451.174  BIC:              -746.943 
log(det(Omega_hat)):     -11.369  Free parameters:        27 
-----------------------------------------------------------------------------------------------------
    Fractional parameters:                                                                             
-----------------------------------------------------------------------------------------------------
    Coefficient              	 Estimate              	  Standard error 
-----------------------------------------------------------------------------------------------------
         d                   	    0.569              	        0.049                
-----------------------------------------------------------------------------------------------------
-----------------------------------------------------------------------------------------------------
    Cointegrating equations (beta):                                                                  
-----------------------------------------------------------------------------------------------------
      Variable        CI equation 1  
-----------------------------------------------------------------------------------------------------
        Var1              1.000     
        Var2              0.111     
        Var3             -0.240     
-----------------------------------------------------------------------------------------------------
Note: Identifying restriction imposed.                                                               
-----------------------------------------------------------------------------------------------------
    Adjustment matrix (alpha):                                                                         
-----------------------------------------------------------------------------------------------------
      Variable        CI equation 1  
-----------------------------------------------------------------------------------------------------
        Var 1            -0.180     
         SE 1         (   0.064  )  
        Var 2             0.167     
         SE 2         (   0.194  )  
        Var 3             0.037     
         SE 3         (   0.014  )  
-----------------------------------------------------------------------------------------------------
Note: Standard errors in parenthesis.                                                                
-----------------------------------------------------------------------------------------------------
    Long-run matrix (Pi):                                                                       
-----------------------------------------------------------------------------------------------------
      Variable         Var 1          Var 2          Var 3   
-----------------------------------------------------------------------------------------------------
      Var 1           -0.180         -0.020          0.043    
      Var 2            0.167          0.019         -0.040    
      Var 3            0.037          0.004         -0.009    
-----------------------------------------------------------------------------------------------------

-----------------------------------------------------------------------------------------------------
    Level parameter (mu):                                                                         
-----------------------------------------------------------------------------------------------------
        Var 1            -0.345     
         SE 1         (   0.069  )  
        Var 2            11.481     
         SE 2         (   0.548  )  
        Var 3            -2.872     
         SE 3         (   0.033  )  
-----------------------------------------------------------------------------------------------------
Note: Standard errors in parenthesis (from numerical Hessian) but asymptotic distribution is unknown. 
-----------------------------------------------------------------------------------------------------
    Lag matrix 1 (Gamma_1):                                                                            
-----------------------------------------------------------------------------------------------------
      Variable         Var 1          Var 2          Var 3   
-----------------------------------------------------------------------------------------------------
      Var 1            0.276         -0.032         -0.510    
       SE 1        (   0.160  )   (   0.026  )   (   0.513  )  
      Var 2           -0.148          1.126         -3.285    
       SE 2        (   0.378  )   (   0.196  )   (   1.975  )  
      Var 3           -0.052          0.008          0.711    
       SE 3        (   0.022  )   (   0.005  )   (   0.170  )  
-----------------------------------------------------------------------------------------------------
Note: Standard errors in parenthesis.                                                                
-----------------------------------------------------------------------------------------------------
    Lag matrix 2 (Gamma_2):                                                                            
-----------------------------------------------------------------------------------------------------
      Variable         Var 1          Var 2          Var 3   
-----------------------------------------------------------------------------------------------------
      Var 1            0.566          0.106          0.609    
       SE 1        (   0.182  )   (   0.045  )   (   0.612  )  
      Var 2            0.493         -0.462          0.450    
       SE 2        (   0.562  )   (   0.198  )   (   2.627  )  
      Var 3           -0.039         -0.020          0.318    
       SE 3        (   0.032  )   (   0.008  )   (   0.143  )  
-----------------------------------------------------------------------------------------------------
Note: Standard errors in parenthesis.                                                                
-----------------------------------------------------------------------------------------------------

-----------------------------------------------------------------------------------------------------
    Roots of the characteristic polynomial                                                           
-----------------------------------------------------------------------------------------------------
    Number     Real part    Imaginary part       Modulus                                             
-----------------------------------------------------------------------------------------------------
       1         -2.893          0.000            2.893                                        
       2         -1.522          0.000            1.522                                        
       3          1.010          0.927            1.371                                        
       4          1.010         -0.927            1.371                                        
       5          1.108          0.000            1.108                                        
       6          1.000          0.000            1.000                                        
       7          1.000         -0.000            1.000                                        
       8          0.944          0.261            0.980                                        
       9          0.944         -0.261            0.980                                        
-----------------------------------------------------------------------------------------------------


-----------------------------------------------------------------------------------------------------
Restrictions imposed on the following parameters:
- Psi. For details see "options.R_psi"
-----------------------------------------------------------------------------------------------------
\end{verbatim}

At the end of the output, a notice is printed to remind the user that restrictions were imposed on \verb|Psi|, i.e.\ on $(d,b)$. In this case, this is the restriction $d=b$ imposed via \verb|opt.restrictDB = 1|.

In addition to the coefficient estimates, we are also interested in testing the model residuals for serial correlation. The results of the white noise tests, called in the last line of Listing \ref{unrEst}, are shown below. For each residual both the Q- and LM-test statistics and their $P$-values are reported, in addition to the multivariate Q-test and $P$-value in the first line of the table. From the output of this table we can conclude that there does not appear to be any problems with serial correlation in the residuals.

\begin{verbatim}
       White Noise Test Results (lag = 12)
---------------------------------------------
Variable |       Q  P-val |      LM  P-val  |
---------------------------------------------
Multivar |  97.868  0.747 |     ----  ----  |
Var1     |   9.301  0.677 |  11.238  0.509  |
Var2     |  14.443  0.273 |   8.566  0.739  |
Var3     |  10.596  0.564 |  12.269  0.424  |
---------------------------------------------
\end{verbatim}

Because \verb|opt.plotRoots = 1| in Listing~\ref{intl2}, the roots of the characteristic polynomial is also plotted along with the unit circle and the transformed unit circle, $\mathbb{C}_{\hat{b}}$, see \cite{Johansen2008}. The plot is shown in Figure~\ref{fig:Roots}.
% Note that the axes of the plot are fixed, and therefore very large roots may not be shown in the plot (in this example, the real root \verb|-2.893|). This should not be a problem since such roots will always be well outside the transformed unit circle.

\begin{figure}[H]
  \centering
  \caption{Roots of characteristic polynomial}
  \label{fig:Roots}
  \includegraphics[scale = .6, keepaspectratio=true]{roots.png}
\end{figure}

Furthermore, the estimation was performed with the grid search and the plot option selected, i.e.\ with \verb|opt.gridSearch = 1| and \verb|opt.plotLike = 1|, which produces a plot of the log-likelihood. The plot for this model is shown in Figure~\ref{fig:m1_likelihood}.

\begin{figure}[H]
  \centering
  \caption{Plot of log-likelihood}
  \label{fig:m1_likelihood}
  \includegraphics[scale = .6, keepaspectratio=true]{m1_likelihood.png}
\end{figure}

The complete results for the unrestricted model are stored in the Matlab structure \verb|m1| and can be accessed anytime. For instance, if the user would like to perform a more careful analysis of the residuals they are stored in \verb|m1.Residuals|.

\subsection{Hypothesis testing}
\label{subsec hypo}

We now move into the hypothesis testing section of the code where we can test several restricted models and perform inference. For restricted model estimation the grid search option is switched off because computation can be very slow, especially in the presence of the level parameter. However, if the user wishes to verify the accuracy of the results or if estimates are close to the upper or lower bound, the grid search option can resolve these issues and give the user additional insight about the behaviour of the likelihood.

All hypotheses are defined as shown in \eqref{R psi}--\eqref{R beta}. The first hypothesis test is $\mathscr{H}_d^1$ (for precise definitions of each hypothesis, please see \cite{JNP2014}), and it is shown in Listing~\ref{Hdb1}. 

\begin{lstlisting}[frame=single,caption={Hypothesis $\mathscr{H}_d^1$}, label = Hdb1]
%% --------- IMPOSE RESTRICTIONS AND TEST THEM ---------- %

DefaultOpt.gridSearch = 0;	% turn off grid search for restricted models
				% because it's too intensive.

%% Test restriction that d=b=1.
opt1 = DefaultOpt;
opt1.R_psi = [1 0];
opt1.r_psi = 1;

m1r1 = FCVARestn(x1, k, r, opt1); % This restricted model is now in the structure 
                                  % m1r1.

mv_wntest(m1r1.Residuals, order, printWNtest);

Hdb = HypoTest(m1, m1r1); % Test the null of m1r1 against the alternative m1 and
			  % store the results in the structure Hdb.
\end{lstlisting}

Here we test the CVAR model (null hypothesis $d=b=1$) against the FCVAR model (alternative hypothesis $d=b\ne 1$). Since \verb|opt1.restrictDB=1| was selected in the choice of options in Listing \ref{intl2}, the restriction that $d=b$ is already imposed. Thus, the user needs to only impose an additional restriction that either $d$ or $b$ is equal to one. In this example, the restriction that $d=1$ is imposed by setting \verb|opt1.R_psi = [1 0]| and \verb|opt1.r_psi = 1|, but the result would be the same if $b=1$ were imposed instead. The restricted model is then estimated and the results are stored in the Matlab structure \verb|m1r1|. As before, the user can perform a series of white noise tests on the residuals by calling the \verb|mv_wntest.m| function. The next step is to perform the actual test. With the results structures from the restricted and unrestricted models, the user can call the function \verb|HypoTest.m| and perform an LR test. This function takes the two model result structures as inputs, automatically compares the number of free parameters to obtain the degrees of freedom, computes the LR test statistic, and displays the output. The results of this test are then stored in the Matlab structure \verb|Hdb| and can be accessed at any time.

Since the output of the estimated model and the white noise tests are similar to the previous example, we only show the output from the hypothesis test. 

\begin{verbatim}
Unrestricted log-likelihood: 451.174
Restricted log-likelihood:   442.027
Test results (df = 1):
LR statistic: 	 18.295
P-value: 	 0.000
\end{verbatim}

The log-likelihoods from both models are reported, along with the degrees of freedom, the LR test statistic, and its $P$-value. In this case the test clearly rejects the null hypothesis that the model is a CVAR. For more significant digits, or to access any of these values from the command window, the user can type \verb|Hdb|.

The next hypothesis of interest is $\mathscr{H}_{\beta}^1$, which is a zero restriction on the first element of the cointegration vector. 

\begin{lstlisting}[frame=single,caption={Hypothesis $\mathscr{H}_{\beta}^1$}, label = Hb1]
%% Test restriction that political variables do not enter the cointegrating relation(s).
opt1 = DefaultOpt;
opt1.R_Beta = [1 0 0];

m1r2 = FCVARestn(x1, k, r, opt1); % This restricted model is now 
                                  % in the structure m1r2.

mv_wntest(m1r2.Residuals, order, printWNtest);

Hbeta1 = HypoTest(m1, m1r2); 	% Test the null of m1r2 against the alternative m1 
				% and store the results in the structure Hbeta1.
\end{lstlisting}

Since the object \verb|opt1| has the restriction $d=b=1$ stored, the first step is to reset the options to default. The restriction on $\beta$ is then specified as in \eqref{R beta}. There are two things to note here. First, the column length of $R_{\beta}$ must equal $p_1 r$, where $p_1=p+1$ if a restricted constant is present and $p_1=p$ otherwise; recall that $p$ is the number of variables in the system and $r$ is the number of cointegrating vectors. Second, zero restrictions are the default and automatically imposed when $r_{\beta}$ is empty. Therefore, the user only needs to specify $r_{\beta}$ if it includes non-zero elements. Recall that for restrictions on $\alpha$ only $r_\alpha = 0$ is allowed so that there is no need to specify $r_\alpha$. As before, the restricted model is estimated with results stored in \verb|m1r2|, the residuals are tested for white noise, and the model under the null is tested against the unrestricted model \verb|m1| with results stored in \verb|Hbeta1|.

Again, since the estimation output is similar to the first example, we only show the results of the hypothesis test here. With a $P$-value close to zero, this hypothesis is also strongly rejected.

\begin{verbatim}
Unrestricted log-likelihood: 451.174
Restricted log-likelihood:   444.395
Test results (df = 1):
LR statistic: 	 13.557
P-value: 	 0.000
\end{verbatim}

Next, we move to tests on $\alpha$. 

\begin{lstlisting}[frame=single,caption={Hypothesis $\mathscr{H}_{\alpha}^1$}, label = Ha1]
%% Test restriction that political variable is long-run exogenous.
opt1 = DefaultOpt;
opt1.R_Alpha = [1 0 0];

m1r3 = FCVARestn(x1, k, r, opt1); % This restricted model is now in the structure 
                                  % m1r3.

mv_wntest(m1r3.Residuals, order, printWNtest);

Halpha1 = HypoTest(m1, m1r3); 	% Test the null of m1r3 against the alternative m1 
				% and store the results in the structure Halpha1.
\end{lstlisting}

Again we first reset \verb|opt1| to the default options to clear previously imposed restrictions. Note that, if it were the case that we failed to reject $\mathscr{H}_{\beta}^1$ and wanted to leave it imposed while adding a restriction on $\alpha$, we could either omit the first line \verb|opt1 = DefaultOpt;|, or we could replace it with \verb|opt1 = m1r2.options;|. The latter assignment is preferred in this case because it is explicit about which model options we are leaving imposed.

The hypothesis $\mathscr{H}_{\alpha}^1$ is tested in the exact same way as before, only now we are changing the variable $R_{\alpha}$ instead of $R_{\beta}$. The results are shown below and we can see that this hypothesis is also rejected.

\begin{verbatim}
Unrestricted log-likelihood: 451.174
Restricted log-likelihood:   446.086
Test results (df = 1):
LR statistic: 	 10.176
P-value: 	 0.001
\end{verbatim}

We next move to the remaining long-run exogeneity tests, $\mathscr{H}_{\alpha}^2$ and $\mathscr{H}_{\alpha}^3$, shown in Listings \ref{Ha2} and \ref{Ha3}, respectively. The results of the tests are shown below each listing.

\begin{lstlisting}[frame=single,caption={Hypothesis $\mathscr{H}_{\alpha}^2$}, label = Ha2]
%% Test restriction that interest-rate is long-run exogenous.
opt1 = DefaultOpt;
opt1.R_Alpha = [0 1 0];

m1r4 = FCVARestn(x1, k, r, opt1); % This restricted model is now in the 
                                  % structure m1r4.

mv_wntest(m1r4.Residuals, order, printWNtest);

Halpha2 = HypoTest(m1, m1r4); 	% Test the null of m1r4 against the alternative m1 
				% and store the results in the structure Halpha2.
\end{lstlisting}

Output:

\begin{verbatim}
Unrestricted log-likelihood: 451.174
Restricted log-likelihood:   450.857
Test results (df = 1):
LR statistic: 	 0.633
P-value: 	 0.426
\end{verbatim}

\begin{lstlisting}[frame=single,caption={Hypothesis $\mathscr{H}_{\alpha}^3$}, label = Ha3]
%% Test restriction that unemployment is long-run exogenous.
opt1 = DefaultOpt;
opt1.R_Alpha = [0 0 1];
k=2; r=1;
m1r5 = FCVARestn(x1, k, r, opt1); % This restricted model is now in the structure 
                                  % m1r5.

mv_wntest(m1r5.Residuals, order, printWNtest);

Halpha3 = HypoTest(m1, m1r5); 	% Test the null of m1r5 against the alternative m1
				% and store the results in the structure Halpha3.
\end{lstlisting}

Output:

\begin{verbatim}
Unrestricted log-likelihood: 451.174
Restricted log-likelihood:   446.184
Test results (df = 1):
LR statistic: 	 9.979
P-value: 	 0.002
\end{verbatim}

The only hypothesis that we fail to reject is $\mathscr{H}_{\alpha}^2$, under which interest rates are long-run exogenous. Since this is the final restricted model, we provide the full estimation output. Note from the output that $\alpha_2 = 0$ as imposed by the restriction.

\begin{verbatim}
-----------------------------------------------------------------------------------------------------
                      Fractionally Cointegrated VAR: Estimation Results                              
-----------------------------------------------------------------------------------------------------
Dimension of system:       3      Number of observations in sample:          316 
Number of lags:            2      Number of observations for estimation:     316 
Restricted constant:      No      Initial values:                              0
Unrestricted constant:    No      Level parameter:                           Yes
Starting value for d:    0.800    Parameter space for d: (0.010 , 2.000) 
Starting value for b:    0.800    Parameter space for b: (0.010 , 2.000) 
-----------------------------------------------------------------------------------------------------
Cointegrating rank:            1  AIC:              -849.715 
Log-likelihood:          450.857  BIC:              -752.065 
log(det(Omega_hat)):     -11.367  Free parameters:        26 
-----------------------------------------------------------------------------------------------------
    Fractional parameters:                                                                             
-----------------------------------------------------------------------------------------------------
    Coefficient              	 Estimate              	  Standard error 
-----------------------------------------------------------------------------------------------------
         d                   	    0.575              	        0.048                
-----------------------------------------------------------------------------------------------------
-----------------------------------------------------------------------------------------------------
    Cointegrating equations (beta):                                                                  
-----------------------------------------------------------------------------------------------------
      Variable        CI equation 1  
-----------------------------------------------------------------------------------------------------
        Var1              0.994     
        Var2              0.105     
        Var3             -0.181     
-----------------------------------------------------------------------------------------------------
    Adjustment matrix (alpha):                                                                         
-----------------------------------------------------------------------------------------------------
      Variable        CI equation 1  
-----------------------------------------------------------------------------------------------------
        Var 1            -0.189     
         SE 1         (   0.065  )  
        Var 2             0.000     
         SE 2         (   0.000  )  
        Var 3             0.039     
         SE 3         (   0.014  )  
-----------------------------------------------------------------------------------------------------
Note: Standard errors in parenthesis.                                                                
-----------------------------------------------------------------------------------------------------
    Long-run matrix (Pi):                                                                       
-----------------------------------------------------------------------------------------------------
      Variable         Var 1          Var 2          Var 3   
-----------------------------------------------------------------------------------------------------
      Var 1           -0.188         -0.020          0.034    
      Var 2            0.000          0.000          0.000    
      Var 3            0.039          0.004         -0.007    
-----------------------------------------------------------------------------------------------------

-----------------------------------------------------------------------------------------------------
    Level parameter (mu):                                                                         
-----------------------------------------------------------------------------------------------------
        Var 1            -0.310     
         SE 1         (   0.067  )  
        Var 2            11.538     
         SE 2         (   0.553  )  
        Var 3            -2.873     
         SE 3         (   0.033  )  
-----------------------------------------------------------------------------------------------------
Note: Standard errors in parenthesis (from numerical Hessian) but asymptotic distribution is unknown. 
-----------------------------------------------------------------------------------------------------
    Lag matrix 1 (Gamma_1):                                                                            
-----------------------------------------------------------------------------------------------------
      Variable         Var 1          Var 2          Var 3   
-----------------------------------------------------------------------------------------------------
      Var 1            0.269         -0.032         -0.512    
       SE 1        (   0.157  )   (   0.026  )   (   0.507  )  
      Var 2           -0.013          1.115         -3.001    
       SE 2        (   0.345  )   (   0.189  )   (   1.909  )  
      Var 3           -0.053          0.008          0.694    
       SE 3        (   0.022  )   (   0.005  )   (   0.164  )  
-----------------------------------------------------------------------------------------------------
Note: Standard errors in parenthesis.                                                                
-----------------------------------------------------------------------------------------------------
    Lag matrix 2 (Gamma_2):                                                                            
-----------------------------------------------------------------------------------------------------
      Variable         Var 1          Var 2          Var 3   
-----------------------------------------------------------------------------------------------------
      Var 1            0.570          0.104          0.586    
       SE 1        (   0.184  )   (   0.044  )   (   0.606  )  
      Var 2            0.685         -0.371          0.223    
       SE 2        (   0.508  )   (   0.159  )   (   2.509  )  
      Var 3           -0.043         -0.020          0.330    
       SE 3        (   0.032  )   (   0.008  )   (   0.138  )  
-----------------------------------------------------------------------------------------------------
Note: Standard errors in parenthesis.                                                                
-----------------------------------------------------------------------------------------------------

-----------------------------------------------------------------------------------------------------
    Roots of the characteristic polynomial                                                           
-----------------------------------------------------------------------------------------------------
    Number     Real part    Imaginary part       Modulus                                             
-----------------------------------------------------------------------------------------------------
       1         -2.710          0.000            2.710                                        
       2         -1.498          0.000            1.498                                        
       3          1.129          0.939            1.469                                        
       4          1.129         -0.939            1.469                                        
       5          1.098          0.000            1.098                                        
       6          1.000          0.000            1.000                                        
       7          1.000          0.000            1.000                                        
       8          0.934          0.281            0.976                                        
       9          0.934         -0.281            0.976                                        
-----------------------------------------------------------------------------------------------------


-----------------------------------------------------------------------------------------------------
Restrictions imposed on the following parameters:
- Psi. For details see "options.R_psi"
- Alpha. For details see "options.R_Alpha"
-----------------------------------------------------------------------------------------------------


       White Noise Test Results (lag = 12)
---------------------------------------------
Variable |       Q  P-val |      LM  P-val  |
---------------------------------------------
Multivar |  97.665  0.752 |     ----  ----  |
Var1     |   9.084  0.696 |  11.267  0.506  |
Var2     |  14.931  0.245 |   9.338  0.674  |
Var3     |  10.729  0.552 |  12.241  0.426  |
---------------------------------------------
\end{verbatim}

Sometimes it is the case that the model output is not normalized with respect to the user's variable of interest, for example when restrictions are imposed on $\alpha$ or $\beta$. For this reason, we also include a code section that normalizes the output, i.e.\ imposes an identity matrix in the first $r \times r$ block of $\beta$. Of course, this code section should only be executed if it does not interfere with any restrictions imposed on the model.

\begin{lstlisting}[frame=single,caption={Normalizing output}, label = normOut]
%% RESTRICTED MODEL OUTPUT - print normalized beta and alpha for model m1r4.
modelRstrct = m1r4;
G = inv(modelRstrct.coeffs.betaHat(1:r,1:r));
betaHatR = modelRstrct.coeffs.betaHat*G;
% alphaHat is post multiplied by G^{-1} so that pi= a(G^{-1})Gb' = ab'
alphaHatR = modelRstrct.coeffs.alphaHat*inv(G)';

display(betaHatR);
display(alphaHatR);
\end{lstlisting}

As an example of when this feature can be useful, consider model $\mathscr{H}_{\alpha}^2$. In the output above, we notice that the cointegrating vector has not been normalized (because restrictions are imposed). The user assigns the model of interest to the variable \verb|modelRstrct|, in this case \verb|m1r4|, and executes the cell. The output is shown below.

\begin{verbatim}
betaHatR =
    1.0000
    0.1057
   -0.1824
alphaHatR =
   -0.1877
         0
    0.0386
    \end{verbatim}


\section{Additional examples: MoreExamples.m}
\label{sec:examples}

To show some additional functionality of the FCVAR software package, this section contains several other examples, which are based on \cite{JNP2014}, but are not part of that paper.

\subsection{Forecasting}
\label{sec:forecasting}

Listing~\ref{forecast} performs recursive one-step ahead forecasts for each of the variables as well as the equilibrium relation. 

\begin{lstlisting}[frame=single,caption={Forecasting}, label = forecast]
%% --------- FORECAST ---------- %

% Forecast from the final restricted model.
NumPeriods = 12; % forecast horizon set to 12 months ahead.

% Assign the model whose coefficients will be used for forecasting.
modelF = m1r4;

xf = FCVARforecast(x1, modelF, NumPeriods);

% Series including forecast.
seriesF = [x1; xf]; 

% Equilibrium relation including forecasts.
equilF = seriesF*modelF.coeffs.betaHat; 

% Determine the size of the vertical line to delimit data and forecast
%   values.
T = size(x1,1);
yMaxS  = max(max(seriesF));
yMinS  = min(min(seriesF));
yMaxEq = max(max(equilF));
yMinEq = min(min(equilF));

% Plot the results.
figure
subplot(2,1,1);
plot(seriesF), 
title('Series including forecast'), xlabel('t');
line([T T], [yMinS yMaxS], 'Color','k');
subplot(2,1,2);
plot(equilF), 
title('Equilibrium relation including forecasts'), xlabel('t');
line([T T], [yMinEq yMaxEq], 'Color','k');
\end{lstlisting}

The user specifies the forecast horizon (\verb|NumPeriods|) as well as the model (in this case, \verb|modelF = m1r1|). These two inputs, along with the data, are used in the call to the function \verb|FCVARforecast.m|. This function returns \verb|xf|, a \verb|NumPeriods| by $p$ matrix of forecasted values of $X$. This code section also plots the original series and the equilibrium relation along with the forecasts. These plots are shown in Figure~\ref{fig:forecast}.

The forecasts can be printed to screen by typing \verb|xf| in the command window. For this example, the forecast yields the following output:
\begin{verbatim}
xf =

   -0.143651232445609   5.857045691984009  -2.636093400885282
   -0.084880236401301   5.959876423543075  -2.654427734412809
   -0.025317647188555   6.110372247993035  -2.673504395263888
    0.023637193159649   6.291703833118448  -2.692354312615941
    0.065948385719753   6.495095415827381  -2.710793696483187
    0.101480482760750   6.712274724017515  -2.728550463910498
    0.131038772676406   6.937036380068601  -2.745463810349039
    0.155108028894535   7.164423215402036  -2.761407736878001
    0.174198620780072   7.390568568007144  -2.776296703046935
    0.188773441631705   7.612444943293763  -2.790074682226306
    0.199275426254352   7.827704702527144  -2.802710965769232
    0.206124920473081   8.034550676278421  -2.814195673339702
\end{verbatim}

\begin{figure}[H]
  \centering
  \caption{Forecast of final model 12 steps ahead}
  \label{fig:forecast}
  % \includegraphics[scale = 1, keepaspectratio=true]{forecast.pdf}
  \includegraphics[scale = 1, keepaspectratio=true]{forecast.png}
\end{figure}

\subsection{Bootstrap hypothesis test}
\label{sec:bootstr-hypoth-test}

Listing~\ref{BSh} demonstrates the use of the wild bootstrap for hypothesis tests on the parameters, as developed by \cite{Boswijk2013} for the CVAR model. The user specifies two sets of options corresponding to two different nested models, and the function \verb|FCVARboot.m| returns the results of the wild bootstrap. The wild bootstrap is programmed to perform under parallel processing. If the user has the capability to use multiple processors, then computation time can be greatly reduced. If not, the function can still be performed, but the bootstrap iterations will appear out of order since the loop is coded using \verb|parfor| instead of \verb|for|.

\begin{lstlisting}[frame=single,caption={Bootstrap hypothesis test}, label = BSh]
%% --------- BOOTSTRAP HYPOTHESIS TEST ---------- %

% Test restriction that political variables do not enter the  
%   cointegrating relation(s).

% Turn off plots for bootstrapping.
DefaultOpt.plotRoots = 0;

% Define estimation options for unrestricted model (alternative)
optUNR = DefaultOpt;

% Define estimation options for restricted model (null)
optRES = DefaultOpt;
optRES.R_Beta = [1 0 0];

% Number of bootstrap samples to generate
B = 999;

% Call to open the distributed processing (comment out if unavailable)
% matlabpool ('open',4); % for versions 2013a and earlier.
% parpool; % for versions 2013b and later.

[LRbs, H, mBS, mUNR] = FCVARboot(x1, k, r, optRES, optUNR, B);


%% Compare the bootstrap distribution to chi-squared distribution

% Estimate kernel density
[F,XI]=ksdensity(LRbs);

% Plot bootstrap density with chi-squared density
figure; plot(XI,F, XI, chi2pdf(XI,H.df))

legend(['Bootstrap PDF with ', num2str(B), ' BS samples'],...
    ['Chi Squared with ', num2str(H.df),' df'])
\end{lstlisting}

An example output 
\begin{verbatim}
Bootstrap results:
Unrestricted log-likelihood: 451.174
Restricted log-likelihood:   444.395
Test results (df = 1):
LR statistic: 	 13.557
P-value: 	      0.000
P-value (BS): 	 0.014
\end{verbatim}

The user might also be interested in comparing the bootstrap likelihood ratio test statistic distribution to the asymptotic one. The second part of Listing~\ref{BSh} performs this comparison by producing a plot of the two distributions, shown in Figure~\ref{fig:BS}.

\begin{figure}[tbh]
  \centering
  \caption{Density of bootstrap LR test statistic}
  \label{fig:BS}
  \includegraphics[scale = 1, keepaspectratio=true]{LRdensity_bw045.png}
\end{figure}


\subsection{Bootstrap rank test}
\label{sec:bootstrap-rank-test}

Listing~\ref{BSr} shows how to perform a wild bootstrap rank test, following the methodology of \cite{Cavaliere2010} for the CVAR model. This procedure works in much the same way as the bootstrap hypothesis test described in Section~\ref{sec:bootstr-hypoth-test}. The difference is that, instead of providing two sets of estimation options, the user specifies two different ranks for comparison.

\begin{lstlisting}[frame=single,caption={Bootstrap rank test}, label = BSr]
%% --------- BOOTSTRAP RANK TEST ---------- %

% Test rank 0 against rank 1
r1 = 0;
r2 = 1;

[LR_Rnk, H_Rnk, mBSr1, mBSr2] = FCVARbootRank(x1, k, DefaultOpt, r1, r2, B);

% Compare to P-value based on asymptotic distribution
fprintf('P-value: \t %1.3f\n', rankTestStats.pv(1));

% Close distributed processing (comment out if unavailable)
% matlabpool close;
\end{lstlisting}

The results are printed as
\begin{verbatim}
Bootstrap results:
Unrestricted log-likelihood: 451.174
Restricted log-likelihood:   440.040
Test results:
LR statistic: 	 22.268
P-value (BS): 	 0.033
P-value: 	 0.043
\end{verbatim}

\subsection{Simulation}
\label{sec:simulation}

Finally, Listing~\ref{simFCVAR} shows how to simulate an FCVAR model for a given set of parameters. The user provides data for starting values and a Matlab structure containing model parameters for simulation as well as the number of periods to simulate. The simulated data is generated using Gaussian errors.

\begin{lstlisting}[frame=single,caption={Simulation}, label = simFCVAR]
%% --------- SIMULATION ---------- %

% Simulate the final restricted model, the same one used for forecasting
%   above.

% Number of periods to simulate
T_sim = 100;

% Simulate data
xSim = FCVARsim(x1, modelF, T_sim);

% Plot the results
figure;
plot(xSim)
legend('Support', 'Unemployment', 'Interest rate')
\end{lstlisting}

For the example above, the generated data is shown in Figure~\ref{fig:sim}.

\begin{figure}[tbh]
  \centering
  \caption{Simulated data}
  \label{fig:sim}
  \includegraphics[scale = 1, keepaspectratio=true]{sim.png}
\end{figure}


\section{Software description}
\label{sec programs}

This section describes the individual components of the software package in detail. The main folder contains the following files:

\begin{itemize}
\item \verb|data_JNP2014.csv|
\item \verb|EstOptions.m|
\item \verb|FCVARestn.m|
\item \verb|FCVARforecast.m|
\item \verb|FCVARboot.m|
\item \verb|FCVARbootRank.m|
\item \verb|FCVARsim.m|
\item \verb|HypoTest.m|
\item \verb|LagSelect.m|
\item \verb|mv_wntest.m|
\item \verb|RankTests.m|
\item \verb|replication_JNP2014.m|
\item \verb|MoreExamples.m|
\end{itemize}

There is one data file (\verb|data_JNP2014.csv|), two scripts (\verb|replication_JNP2014.m| and  \verb|MoreExamples.m|), one class definition (\verb|EstOptions.m|), and nine functions. These main functions depend on 17 auxiliary functions stored in the subfolder \verb|Auxiliary|, as well as the functions \verb|extrema.m| and \verb|extrema2.m|, which are also in the subfolder \verb|Auxiliary|. The replication script adds these auxiliary files to the path definition so that the main functions have access to them. 

We remark again that it should not really be necessary to modify any files except the scripts, i.e.\ \verb|replication_JNP2014.m| or \verb|MoreExamples.m|. Only advanced users wishing to modify or extend the actual functionality of the programs will need to make any changes to the remaining files. The following subsections briefly describe the functionality of each program file.


\subsection{EstOptions.m}
\label{sec:estoptions.m}
\lstset{firstnumber=1}

\begin{lstlisting}[frame=single,caption={EstOptions.m}, label = EstOpt]
classdef EstOptions  
% classdef EstOptions 
% Written by Michal Popiel and Morten Nielsen (This version 04.11.2016)
% 
% DESCRIPTION: This class defines the estimation options used in the FCVAR
% 	estimation procedure and the related programs. Assigning this class
% 	to a variable stores the default properties defined below in that
% 	variable. In addition to the properties, the methods section includes the
% 	function updateRestrictions which performs several checks on the
% 	user-specified options prior to estimation.
%_________________________________________________________________________
\end{lstlisting}

\verb|EstOptions| is a class definition which is assigned to an object and used in most of the functions. It contains all the model specifications and options that are available to the user. Here is an example of the contents when \verb|EstOptions| is entered in the command line:

\begin{verbatim}
>> EstOptions

ans = 

  EstOptions with properties:

    UncFminOptions: [1x1 struct]
     ConFminOptions: [1x1 struct]
         LineSearch: 1
           LocalMax: 1
              dbMax: 2
             dbMin: 0.0100
               db0: [1 1]
       constrained: 1
        restrictDB: 1
                 N: 0
       unrConstant: 0
         rConstant: 0
        levelParam: 1
              C_db: []
              c_db: []
             UB_db: []
             LB_db: []
             R_psi: []
             r_psi: []
           R_Alpha: []
           r_Alpha: []
            R_Beta: []
            r_Beta: []
      print2screen: 1
       printGammas: 1
        printRoots: 1
         plotRoots: 1
        gridSearch: 1
          plotLike: 1
          progress: 1
        updateTime: 5
           progLoc: '"/usr/bin/fdpval"'
            CalcSE: 1
\end{verbatim}

\subsection{FCVARestn.m}
\label{subsec fcvarestn}

\begin{lstlisting}[frame=single,caption={FCVARestn.m}, label = FCVARestn]
function [ results ] = FCVARestn(x,k,r,opt)
% function [ results ] = FCVARestn(x,k,r,opt)
% Written by Michal Popiel and Morten Nielsen (This version 04.09.2016)
% 
% DESCRIPTION: This function performs estimation of the FCVAR system. It is
% 	the main function in the program with several nested functions, each
% 	described below. It estimates the model parameters, calculates the
% 	standard errors and the number of free parameters, obtains the residuals
% 	and the roots of the characteristic polynomial, and prints the output.
%
% Input = x (matrix of variables to be included in the system)
%         k (number of lags)
%         r (number of cointegrating vectors)
%         opt (object containing the estimation options)
% Output = results (a Matlab structure containing estimation results)
%            - results.startVals     (Starting values used for optimization)
%            - results.options       (Estimation options)
%            - results.like          (Model log-likelihood)
%            - results.coeffs        (Parameter estimates)
%            - results.rankJ         (Rank of Jacobian for 
%					identification condition)
%            - results.fp            (Number of free parameters)
%            - results.SE            (Standard errors)
%            - results.NegInvHessian (Negative of inverse Hessian matrix)
%            - results.Residuals     (Model residuals)
%            - results.cPolyRoots    (Roots of characteristic polynomial)
%_________________________________________________________________________
\end{lstlisting}

This function is the central estimation function in the program. Calling this function returns a ``results'' Matlab structure. An example of a typical results structure is shown here:
\begin{verbatim}
m1r4 = 
        startVals: [0.8000 0.8000 -0.0938 11.5400 -2.8633]
          options: [1x1 EstOptions]
             like: 450.8574
           coeffs: [1x1 struct]
            rankJ: 4
               fp: 26
               SE: [1x1 struct]
    NegInvHessian: [25x25 double]
        Residuals: [316x3 double]
       cPolyRoots: [9x1 double]
\end{verbatim}


\subsection{LagSelect.m}

\begin{lstlisting}[frame=single,caption={LagSelect.m}]
function  LagSelect(x, kmax, r, order, opt )
% function LagSelect(x, kmax, r, order, opt )
% Written by Michal Popiel and Morten Nielsen (This version 3.31.2016)
% 
% DESCRIPTION: This program takes a matrix of variables and performs lag
% 	selection on it by using the likelihood ratio test. Output and test
% 	results are printed to the screen.
%
% Input = x     (matrix of variables to be included in the system)
%         kmax  (maximum number of lags)
%         r     (cointegration rank = number of cointegrating vectors)
%         order (order of serial correlation for white noise tests)
%         opt   (object containing estimation options)
% Output = none (only output to screen)
%_________________________________________________________________________
\end{lstlisting}


\subsection{RankTests.m}
\begin{lstlisting}[frame=single,caption={RankTests.m}]
function [ rankTestStats ] = RankTests(x, k, opt)
% function [ rankTestStats ] = RankTests(x, k, opt)
% Written by Michal Popiel and Morten Nielsen (This version 11.17.2014)
% Based on Lee Morin & Morten Nielsen (June 5, 2013)
%
% DESCRIPTION: Performs a sequence of  likelihood ratio tests 
% 	for cointegrating rank.
% 
% The results are printed to screen if the indicator print2screen is 1.
%
% input = vector or matrix x of data.
%       scalar k denoting lag length.
%       opt (object containing estimation options)
% 
% output = rankTestStats structure with results from cointegrating rank 
%           tests, containing the following (p+1) vectors with i'th element
%           corresponding to rank = i-1:
%	dHat	(estimates of d)
%	bHat	(estimate of b)
%	LogL	(maximized log-likelihood)
%	LRstat  (LR trace statistic for testing rank r against rank p)
%	pv      (P-value of LR trace test, or "999" if P-value is not available)
%______________________________________________________
\end{lstlisting}%

\subsubsection{get\textunderscore{pvalues}}
\label{sec:getpvals}
\begin{lstlisting}[frame=single,caption={get\textunderscore{pvalues}.m}]
function [pv] = get_pvalues(q, b, consT, testStat, opt)
% Written by Michal Popiel and Morten Nielsen (This version 10.22.2014)
% 
% DESCRIPTION: This function calls the program FDPVAL in the terminal and
% returns the P-value based on the user's inputs. The function's 
% arguments must be converted to strings in order to interact with the
% terminal. 
% 
% Input = q        (number of variables minus rank)
% b        (parameter)
% consT    (boolean variable indicating whether or not there is
% constant present)
% testStat (value of the test statistic)
% opt (object containing estimation options)
% Output = pv (P-value for likelihood ratio test)
% _________________________________________________________________________
\end{lstlisting}


\subsection{HypoTest.m}

\begin{lstlisting}[frame=single,caption={HypoTest.m}]
function results = HypoTest(modelUNR, modelR)
% function results = HypoTest(modelUNR, modelR)
% Written by Michal Popiel and Morten Nielsen (This version 2.24.2015)
% 
% DESCRIPTION: This function performs a likelihood ratio test of the null
% 	hypothesis: "model is modelR" against the alternative hypothesis:
% 	"model is modelUNR".
%
% Input = modelUNR (structure of estimation results created for unrestricted model)
%         modelR (structure of estimation results created for restricted model)
% Output = results: a Matlab structure containing test results
%            - results.loglikUNR (loglikelihood of unrestricted model)
%            - results.loglikR   (loglikelihood of restricted model)
%            - results.df        (degrees of freedom for the test)
%            - results.LRstat    (likelihood ratio test statistic)
%            - results.p_LRtest  (P-value for test)
%_________________________________________________________________________
\end{lstlisting}


\subsection{FCVARforecast.m}
\begin{lstlisting}[frame=single,caption={FCVARforecast.m}]
function xf = FCVARforecast(data, model, NumPeriods)
% function xf = FCVARforecast(data, model, NumPeriods)
% Written by Michal Popiel and Morten Nielsen (This version 11.17.2014)
% 
% DESCRIPTION: This function calculates recursive forecasts. It uses
%   FracDiff() and Lbk(), which are nested below.
%
% Input = data (T x p matrix of data)
%         model (a Matlab structure containing estimation results)
%         NumPeriods (number of steps ahead for forecast)
% Output = xf (NumPeriods x p matrix of forecasted values)
%_________________________________________________________________________
\end{lstlisting}


\subsection{mv\textunderscore{wntest.m}}

\begin{lstlisting}[frame=single,caption={mv\textunderscore{wntest.m}}]
function [ Q, pvQ, LM, pvLM, mvQ, pvMVQ ] = mv_wntest(x, maxlag, printResults)
% function [ Q, pvQ, LM, pvLM, mvQ, pvMVQ ] = ...
%                       mv_wntest(x, maxlag, printResults)
% Written by Michal Popiel and Morten Nielsen (This version 7.21.2015)
% 
% DESCRIPTION: This function performs a multivariate Ljung-Box Q-test for  
% 	white noise and univariate Q-tests and LM-tests for white noise on the 
% 	columns of x.
% 	The LM test should be consistent for heteroskedastic series, Q-test is not. 
%
% Input = x            (matrix of variables under test, typically model residuals)
%         maxlag       (number of lags for serial correlation tests)
%         printResults (set =1 to print results to screen)
% Output = Q     (1xp vector of Q statistics for individual series)
%		   pvQ 	 (1xp vector of P-values for Q-test on individual series)
%          LM    (1xp vector of LM statistics for individual series)
%		   pvLM	 (1xp vector of P-values for LM-test on individual series)
%          mvQ   (multivariate Q statistic)
%          pvMVQ (P-value for multivariate Q-statistic using p^2*maxlag df)
%_________________________________________________________________________
\end{lstlisting}

\subsubsection{LMtest}
\begin{lstlisting}[frame=single,caption={LMtest.m}]
function [ LMstat, pv ] = LMtest(x,q)
% Breusch-Godfrey Lagrange Multiplier test for serial correlation.
\end{lstlisting}

\subsubsection{Qtest}
\begin{lstlisting}[frame=single,caption={Qtest.m}]
function [ Qstat, pv ] = Qtest(x, maxlag)
% (Multivariate) Ljung-Box Q-test for serial correlation, see 
% 	Luetkepohl (2005, New Introduction to Multiple Time Series Analysis, p. 169).
\end{lstlisting}

\subsection{FCVARboot.m}
The wild bootstrap procedure for hypothesis tests on the parameters is based on the procedure for the I(1) model in \cite{Boswijk2013}.
\begin{lstlisting}[frame=single,caption={FCVARboot.m}]
function [LRbs, H, mBS, mUNR] = FCVARboot(x, k, r, optRES, optUNR, B)
% function [LRbs, H, mBS, mUNR] = FCVARboot(x, k, r, optRES, optUNR, B)
% Written by Michal Popiel and Morten Nielsen (This version 08.06.2015)
% 
% DESCRIPTION: This function generates a distribution of a likelihood ratio
%           test statistic using a Wild bootstrap, following the method of
%			Boswijk, Cavaliere, Rahbek, and Taylor (2013). It takes two sets 
%           of options as inputs to estimate the model under the null and the
%           unrestricted model. 
%
% Input = x      (data - if k>0, actual data is used for initial values)
%         k      (number of lags)
%         optRES (options object for restricted model under the null)
%         optUNR (options object to estimate unrestricted model)
%         B      (number of bootstrap samples)
% 
% Output = LRbs (B x 1 vector simulated likelihood ratio statistics)
%          pv   (approximate p-value for LRstat based on bootstrap 
%                                                       distribution)
%          H    (a Matlab structure containing LR test results, it is
%               identical to the output from HypoTest, with one addition,
%               namely H.pvBS which is the Bootstrap P-value)
%          mBS  (model estimates under the null)
%          mUNR (model estimates under the alternative)
%_________________________________________________________________________
\end{lstlisting}

\subsection{FCVARbootRank.m}
The wild bootstrap procedure for hypothesis tests on the parameters is based on the procedure for the I(1) model in \cite{Cavaliere2010}.
\begin{lstlisting}[frame=single,caption={FCVARbootRank.m}]
function [LRbs, H, mBS, mUNR] = FCVARbootRank(x, k, opt, r1, r2, B)
% function [LRbs, H, mBS, mUNR] = FCVARbootRank(x, k, opt, r1, r2, B)
% Written by Michal Popiel and Morten Nielsen (This version 08.06.2015)
% 
% DESCRIPTION: This function generates a distribution of a likelihood ratio
%           test statistic for the rank test using a Wild bootstrap, 
%			following the method of Cavaliere, Rahbek, and Taylor (2010). It 
%           takes the two ranks as inputs to estimate the model under the 
%           null and the model under the alternative.
%
% Input = x  (data - if k>0, actual data is used for initial values)
%         k  (number of lags)
%		  opt(estimation options)
%         r1 (rank under the null)
%         r2 (rank under the alternative)
%         B  (number of bootstrap samples)
% 
% Output = LRbs (B x 1 vector simulated likelihood ratio statistics)
%          pv (approximate p-value for LRstat based on bootstrap 
%                                                       distribution)
%          H (a Matlab structure containing LR test results, it is
%               identical to the output from HypoTest, with one addition,
%               namely H.pvBS which is the Bootstrap P-value)
%          mBS  (model estimates under the null)
%          mUNR (model estimates under the alternative)
%_________________________________________________________________________
\end{lstlisting}

\subsection{FCVARsim.m}
\begin{lstlisting}[frame=single,caption={FCVARsim.m}]
function xSim = FCVARsim(data, model, NumPeriods)
% function xSim = FCVARsim(data, model, NumPeriods)
% Written by Michal Popiel and Morten Nielsen (This version 08.06.2015)
% 
% DESCRIPTION: This function simulates the FCVAR model as specified by
%               input "model" and starting values specified by "data."
%               Errors are drawn from a Normal distribution. 
%
% Input = data (T x p matrix of data)
%         model (a Matlab structure containing estimation results)
%         NumPeriods (number of steps for simulation)
% Output = xSim (NumPeriods x p matrix of simulated values)
%_________________________________________________________________________
\end{lstlisting}


\subsection{Auxillary functions}

\subsubsection{FCVARsimBS.m}
\begin{lstlisting}[frame=single,caption={FCVARsim.m}]
function xBS = FCVARsimBS(data, model, NumPeriods)
% function xBS = FCVARsimBS(data, model, NumPeriods)
% Written by Michal Popiel and Morten Nielsen (This version 08.06.2015)
% 
% DESCRIPTION: This function simulates the FCVAR model as specified by
%               input "model" and starting values specified by "data." It
%               creates a bootstrap sample by augmenting each iteration
%               with a bootstrap error. The errors are sampled from the
%               residuals specified under the "model" input and have a
%               positive or negative sign with equal probability
%               (Rademacher distribution). 
%
% Input = data       (T x p matrix of data)
%         model      (a Matlab structure containing estimation results)
%         NumPeriods (number of steps for simulation)
% Output = xBS       (NumPeriods x p matrix of simulated bootstrap values)
%_________________________________________________________________________
\end{lstlisting}


\subsubsection{FCVARhess.m}
\begin{lstlisting}[frame=single,caption={FCVARhess.m}]
function [ hessian ] = FCVARhess(x, k, r, coeffs, opt)
% function [ hessian ] = FCVARhess(x, k, r, coeffs, opt)
% Written by Michal Popiel and Morten Nielsen (This version 10.22.2014)
% 
% DESCRIPTION: This function calculates the Hessian matrix of the
% 	log-likelihood numerically.
%
% Input = x (matrix of variables to be included in the system)
%         k (number of lags)
%         r (number of cointegrating vectors)
%         coeffs (coefficient estimates around which estimation takes place)
%         opt (object containing the estimation options)
% Output = hessian (matrix of second derivatives)
%_________________________________________________________________________
\end{lstlisting}

\subsubsection{FCVARlike.m}
\begin{lstlisting}[frame=single,caption={FCVARlike.m}]
function [ like ] = FCVARlike(x, params, k, r, opt)
% function [ like ] = FCVARlike(x, params, k, r, opt)
% Written by Michal Popiel and Morten Nielsen (This version 11.10.2014)
% 
% DESCRIPTION: This function adjusts the variables with the level parameter,
% 	if present, and returns the log-likelihood given d,b.
%
% Input = x   (matrix of variables to be included in the system)
%         params (a vector of parameters d,b, and mu (if option selected))
%         k   (number of lags)
%         r   (number of cointegrating vectors)
%         opt (object containing the estimation options)
% Output = like (concentrated log-likelihood evaluated at given parameters)
%_________________________________________________________________________
\end{lstlisting}

\subsubsection{FCVARlikeMu.m}
\begin{lstlisting}[frame=single,caption={FCVARlikeMu.m}]
function [ like ] = FCVARlikeMu(y, db, mu, k, r, opt)
% function [ like ] = FCVARlikeMu(y, db, mu, k, r, opt)
% Written by Michal Popiel and Morten Nielsen (This version 10.22.2014)
% 
% DESCRIPTION: This function evaluates the likelihood for a given set of
% 	parameter values. It is used by the LikeGrid() function to numerically
% 	optimize over the level parameter for given values of the fractional
% 	parameters.
%
% Input = y  (matrix of variables to be included in the system)
%         db (fractional parameters d,b)
%         mu (level parameter)
%         k  (number of lags)
%         r  (number of cointegrating vectors)
%         opt (object containing the estimation options)
% Output = like (log-likelihood evaluated at specified parameter values)
%_________________________________________________________________________
\end{lstlisting}


\subsubsection{FracDiff.m}
\begin{lstlisting}[frame=single,caption={FracDiff.m}]
function [dx] = FracDiff(x, d)
    % function [dx] = FracDiff(x,d)
    % Andreas Noack Jensen & Morten Nielsen
    % May 24, 2013
    %
    % FracDiff(x,d) is a fractional differencing procedure based on the
    % 	fast fractional difference algorithm of Jensen & Nielsen (2014, JTSA).
    %
    % input = x (vector or matrix of data)
    %         d (scalar value at which to calculate the fractional difference)
    % 
    % output = vector or matrix (1-L)^d x of same dimensions as x.
    %______________________________________________________
\end{lstlisting}

The function \verb|FracDiff.m| is the implementation of the fast fractional difference algorithm by \cite{Jensen2014}.
 

\subsubsection{FreeParams.m}
\begin{lstlisting}[frame=single,caption={FreeParams.m}]
function [ fp ] = FreeParams(k, r, p, opt, rankJ)
% function [ fp ] = FreeParams(k, r, p, opt, rankJ)
% Written by Michal Popiel and Morten Nielsen (This version 10.22.2014)
% 
% DESCRIPTION: This function counts the number of free parameters based on
% 	the number of coefficients to estimate minus the total number of
% 	restrictions. When both alpha and beta are restricted, the rank condition
% 	is used to count the free parameters in those two variables.
%
% Input = x (matrix of variables to be included in the system)
%         k (number of lags)
%         r (number of cointegrating vectors)
%         opt (object containing the estimation options)
% Output = fp (number of free parameters)
%_________________________________________________________________________
\end{lstlisting}

\subsubsection{FullFCVARlike.m}
\begin{lstlisting}[frame=single,caption={FullFCVARlike.m}]
function [ like ] = FullFCVARlike(x, k, r, coeffs, beta, rho, opt)
% function [ like ] = FullFCVARlike(x, k, r, coeffs, beta, rho, opt)
% Written by Michal Popiel and Morten Nielsen (This version 10.22.2014)
% Based on Lee Morin & Morten Nielsen (August 22, 2011)
% 
% DESCRIPTION: This function returns the value of the log-likelihood
% 	evaluated at the parameters provided as inputs.
%
% Input = x (matrix of variables to be included in the system)
%         k (number of lags)
%         r (number of cointegrating vectors)
%         coeffs (Matlab structure of coefficients)
%         beta (value of beta)
%         rho  (value of rho)
%         opt  (object containing the estimation options)
% Output = like (value of the log likelihood)
%_________________________________________________________________________
\end{lstlisting}

\subsubsection{GetParams.m}
\begin{lstlisting}[frame=single,caption={GetParams.m}]
function [ estimates ] = GetParams(x, k, r, db, opt)
% function [ estimates ] = GetParams(x, k, r, db, opt)
% Written by Michal Popiel and Morten Nielsen (This version 04.13.2016)
% Based on Lee Morin & Morten Nielsen (August 22, 2011)
% 
% DESCRIPTION: This function uses FWL and reduced rank regression to obtain
% the estimates of Alpha, Beta, Rho, Pi, Gamma, and Omega
%
% Input = x   (matrix of variables to be included in the system)
%         k   (number of lags)
%         r   (number of cointegrating vectors)
%         db  (value of d and b)
%         opt (object containing the estimation options)
% Output = estimates (Matlab structure containing the following)
%          - estimates.db (taken directly from the input)
%          - estimates.alphaHat
%          - estimates.betaHat
%          - estimates.rhoHat
%          - estimates.piHat
%          - estimates.OmegaHat
%          - estimates.GammaHat ( p x kp matrix [GammaHat1,...,GammaHatk])
%_________________________________________________________________________
\end{lstlisting}

\subsubsection{GetResiduals.m}
\begin{lstlisting}[frame=single,caption={GetResiduals.m}]
function [ epsilon ] = GetResiduals(x, k, r, coeffs, opt)
% function [ epsilon ] = GetResiduals(x, k, r, coeffs, opt)
% Written by Michal Popiel and Morten Nielsen (This version 10.22.2014)
% Based on Lee Morin & Morten Nielsen (August 22, 2011)
% 
% DESCRIPTION: This function calculates the model residuals.
%
% Input = x (matrix of variables to be included in the system)
%         k (number of lags)
%         r (number of cointegrating vectors)
%         coeffs (Matlab structure of coefficients)
%         opt (object containing the estimation options)
% Output = epsilon (matrix of residuals from model estimation evaluated at
%                     the parameter estimates specified in coeffs) 
%_________________________________________________________________________
\end{lstlisting}

\subsubsection{Lbk.m}
\begin{lstlisting}[frame=single,caption={Lbk.m}]
function [ Lbkx ] = Lbk(x, b, k)
% function [ Lbkx ] = Lbk(x, b, k)
% Written by Michal Popiel and Morten Nielsen (This version 10.22.2014)
% Based on Lee Morin & Morten Nielsen (May 24, 2013)
%
% DESCRIPTION: Lbk(x, b, k) is a lag polynomial in the fractional lag operator.
%
% Input = x (vector or matrix of data)
%         b (scalar value at which to calculate the fractional lag)
%         k (number of lags)
% 
% Output = matrix  [ Lb^1 x, Lb^2 x, ..., Lb^k x] where Lb = 1 - (1-L)^b.
%          The output matrix has the same number of rows as x but k times 
%           as many columns.
% 
% Calls the function FracDiff(x, d) 
%______________________________________________________
\end{lstlisting}

\subsubsection{LikeGrid.m}

\begin{lstlisting}[frame=single,caption={LikeGrid.m}]
function [ params ] = LikeGrid(x,k,r,opt)
% function [ params ] = LikeGrid(x,k,r,opt)
% Written by Michal Popiel and Morten Nielsen (This version 04.12.2016)
% 
% DESCRIPTION: This function evaluates the likelihood over a grid of values
% 	for (d,b) (or phi). It can be used when parameter estimates are sensitive to 
% 	starting values to give an approximation of the global max which can 
% 	then be used as the starting value in the numerical optimization in 
% 	FCVARestn().
%
% Input = x   (matrix of variables to be included in the system)
%         k   (number of lags)
%         r   (number of cointegrating vectors)
%         opt (object containing the estimation options)
% Output = params (row vector of d,b, and mu (if level parameter is selected)
%					corresponding to a maximum over the grid of (d,b), or phi)
%
% Note:	If opt.LocalMax == 0, LikeGrid returns the parameter values
%       corresponding to the global maximum of the likelihood on the grid.
%       If opt.LocalMax == 1, LikeGrid returns the parameter values for the
%       local maximum corresponding to the highest value of b. This
%       alleviates the identification problem mentioned in Johansen and
%       Nielsen (2010, section 2.3).
%_________________________________________________________________________
\end{lstlisting}

This function allows the user to pre-estimate to obtain starting values by using a grid search. There are four types of estimation that the grid search can perform. If $d$ and $b$ are completely unconstrained, the grid search is over two dimensions within the bounds specified by \verb|opt.dbMin| and \verb|opt.dbMax|. An example of the likelihood obtained in an unconstrained grid search is shown in Figure~\ref{fig:gridFree}. Next, if $d\ge b$ is imposed via \verb|opt.constrained = 1| (imposed in \cite{johansen2012likelihood} but relaxed in \cite{JN2018}) the computation can be cut in half. An example of this likelihood is shown in Figure~\ref{fig:gridConstrained}. If the restriction $d=b$ is imposed, then the grid search is one-dimensional as shown in Figure~\ref{fig:gridDB}. Finally, if a restriction is imposed on either $d$ or $b$ via $R_\psi$ and $r_\psi$ in \eqref{R psi}, then the grid search is also one-dimensional. An example of this situation is shown in Figure~\ref{fig:gridPhi}. Note that the $x$-axis is over the parameter $\phi$ and the fractional parameters are found from
\begin{equation}
  \begin{bmatrix}
    d \\ b
  \end{bmatrix}
  % [d \quad b]
  = H\phi + h,
\end{equation}
where $H = (R_{\psi}^{\prime})_\perp$ and $h = R_{\psi}^{\prime} (R_\psi R_{\psi}^{\prime})^{-1} r_\psi$. The bounds on $\phi$ are derived from \verb|opt.dbMin| and \verb|opt.dbMax| in a similar way.

\begin{figure}[H]
  \label{fig:LikeGrid}
  \centering
  \caption{Grid search}
  \subfigure[Unconstrained]{
    \label{fig:gridFree}
    \includegraphics[scale = 0.5, keepaspectratio=true]{grid3d.png}
  } 
  \subfigure[$d\ge b$]{ 
    \label{fig:gridConstrained}
    \includegraphics[scale = 0.5, keepaspectratio=true]{gridConst.png}
  }
  \\
  \subfigure[$d = b$]{
    \label{fig:gridDB}
    \includegraphics[scale = 0.5, keepaspectratio=true]{gridDB.png}
  } 
  \subfigure[Restrictions imposed]{
    \label{fig:gridPhi}
    \includegraphics[scale = 0.5, keepaspectratio=true]{gridPhi.png}
  }
\end{figure}

\subsubsection{RstrctOptm\textunderscore{}Switch.m}
\begin{lstlisting}[frame=single,caption={RstrctOptm\textunderscore{}Switch.m}]
function [betaStar, alphaHat, OmegaHat] = RstrctOptm_Switch(beta0, S00, S01, S11, T, p, opt)
% function [betaStar, alphaHat, OmegaHat] 
%                   = RstrctOptm_Switch(beta0, S00, S01, S11, T, p, opt)
% Written by Michal Popiel and Morten Nielsen (This version 03.29.2016)
% 
% DESCRIPTION: This function is imposes the switching algorithm of Boswijk
%   and Doornik (2004, page 455) to optimize over free parameters psi 
%   and phi directly, combined with the line search proposed by 
%	Doornik (2018, Section 2.2). We translate between  (psi, phi) and 
%	(alpha, beta) using the relation of R_Alpha*vec(alpha) = 0 and 
%	A*psi = vec(alpha'), and R_Beta*vec(beta) = r_beta and 
%	H*phi+h = vec(beta). Note the transposes.
%
% Input = beta0 (unrestricted estimate of beta)
%         S00, S01, S11 (product moments)
%         T (number of observations)
%         p (number of variables)
%         opt (object containing the estimation options)
% Output = betaStar (estimate of betaStar)
%          alphaHat (estimate of alpha)
%          OmegaHat (estimate of Omega)
%_________________________________________________________________________
\end{lstlisting}

\subsubsection{SEmat2vecU.m}
\begin{lstlisting}[frame=single,caption={SEmat2vecU.m}]
function [ param ] = SEmat2vecU( coeffs, k, r, p , opt)
% function [ param ] = SEmat2vecU( coeffs, k, r, p , opt)
% Written by Michal Popiel and Morten Nielsen (This version 10.22.2014)
% Based on Lee Morin & Morten Nielsen (August 22, 2011)
% 
% DESCRIPTION: This function transforms the model parameters in matrix
% 	form into a vector.
%
% Input = coeffs (Matlab structure of coefficients in their usual matrix form)
%         k (number of lags)
%         r (number of cointegrating vectors)
%         p (number of variables in the system)
%         opt (object containing the estimation options)
% Output = param (vector of parameters)
%_________________________________________________________________________
\end{lstlisting}

\subsubsection{SEvec2matU.m}
\begin{lstlisting}[frame=single,caption={SEvec2matU.m}]
function [ coeffs ] = SEvec2matU( param, k, r, p, opt )
% function [ coeffs ] = SEvec2matU( param, k, r, p, opt )
% Written by Michal Popiel and Morten Nielsen (This version 10.22.2014)
% Based on Lee Morin & Morten Nielsen (August 22, 2011)
% 
% DESCRIPTION: This function transforms the vectorized model parameters
% 	into matrices.
%
% Input = param (vector of parameters)
%         k (number of lags)
%         r (number of cointegrating vectors)
%         p (number of variables in the system)
%         opt (object containing the estimation options)
% Output = coeffs (Matlab structure of coefficients in their usual matrix form)
%_________________________________________________________________________
\end{lstlisting}

\subsubsection{TransformData.m}
\begin{lstlisting}[frame=single,caption={TransformData.m}]
function [ Z0, Z1, Z2, Z3 ] = TransformData(x, k, db, opt)
% function [ Z0, Z1, Z2, Z3 ] = TransformData(x, k, db, opt)
% Written by Michal Popiel and Morten Nielsen (This version 10.22.2014)
% Based on Lee Morin & Morten Nielsen (May 24, 2013)
%
% DESCRIPTION: Returns the transformed data required for regression and
% 	reduced rank regression.
% 
% Input = x   (matrix of variables to be included in the system)
%         k   (number of lags)
%         db  (fractional differencing parameters d and b)
%         opt (object containing the estimation options)
% Output = Z0, Z1, Z2, and Z3 of transformed data.
%
% Calls the function FracDiff(x, d) and Lbk(x, b, k).
% _____________________________________________________
\end{lstlisting}

\subsubsection{CharPolyRoots.m}
\begin{lstlisting}[frame=single,caption={CharPolyRoots.m}]
function cPolyRoots = CharPolyRoots(coeffs, opt, k, r, p)
% function cPolyRoots = CharPolyRoots(coeffs, opt, k, r, p)
% Written by Michal Popiel and Morten Nielsen (This version 12.07.2015)
% Based on Lee Morin & Morten Nielsen (May 31, 2013)
%
% DESCRIPTION: CharPolyRoots calculates the roots of the 
%     characteristic polynomial and plots them with the unit circle 
%     transformed for the fractional model, see Johansen (2008).
% 
% input = coeffs (Matlab structure of coefficients
%         opt (object containing the estimation options)
%         k (number of lags)
%         r (number of cointegrating vectors)
%         p (number of variables in the system)
% 
% output = complex vector cPolyRoots with the roots of the characteristic polynomial.
% 
% No dependencies.
% 
% Note: The roots are calculated from the companion form of the VAR, 
%       where the roots are given as the inverse eigenvalues of the 
%       coefficient matrix.
%______________________________________________________
\end{lstlisting}

\subsubsection{GetBounds.m}
\begin{lstlisting}[frame=single,caption={GetBounds.m}]
function [ UB, LB ] = GetBounds(opt)
% function [ UB, LB ] = GetBounds(opt)
% Written by Michal Popiel and Morten Nielsen (This version 04.11.2016)
% 
% DESCRIPTION: This function obtains upper and lower bounds on d,b or on 
%   phi, given by db = H*phi + h. 
%
% Input  = opt   (object containing estimation options)
% Output = UB (a 2x1 or 1x1 upper bound for db or phi)
%          LB (a 2x1 or 1x1 lower bound for db or phi)
%_________________________________________________________________________
\end{lstlisting}

\subsubsection{FCVARsimBS.m}
\begin{lstlisting}[frame=single,caption={FCVARsimBS.m}]
function xBS = FCVARsimBS(data, model, NumPeriods)
% function xBS = FCVARsimBS(data, model, NumPeriods)
% Written by Michal Popiel and Morten Nielsen (This version 02.09.2015)
% 
% DESCRIPTION: This function simulates the FCVAR model as specified by
%               input "model" and starting values specified by "data." It
%               creates a bootstrap sample by augmenting each iteration
%               with a bootstrap error. The errors are sampled from the
%               residuals specified under the "model" input and have a
%               positive or negative sign with equal probability
%               (Rademacher distribution). 
%
% Input = data       (T x p matrix of data)
%         model      (a Matlab structure containing estimation results)
%         NumPeriods (number of steps for simulation)
% Output = xBS       (NumPeriods x p matrix of simulated bootstrap values)
%_________________________________________________________________________
\end{lstlisting}


\newpage
\appendix
\section{Version change log}
\label{sec version}

\subsection{Version 1.0.0: October 24, 2014}
\noindent First publicly available version.

\subsection{Version 1.1.0: October 30, 2014}
\noindent \verb|FCVARestn.m|
\begin{itemize}
\item Fixed declaration of number of observations $T$ so that it accounts for initial values \verb|opt.N|. This affects AIC, BIC calculations and printed number of observations in the output, but nothing else.
\item Changed number of significant digits in the printed output for likelihood, AIC, BIC.
\item Added redundancy check for restrictions in \verb|R_psi| matrix and \verb|restrictDB|.
\item Changed estimation method when \verb|R_psi| (together with \verb|restrictDB|) has rank $2$.
\end{itemize}

\noindent \verb|LikeGrid.m|
\begin{itemize}
\item Changed output so that actual (restricted) $d,b$ is shown in waitbar/terminal.
\item Fixed how the endpoints for $\phi$ are calculated if \verb|R_psi| is non-empty.
\item Changed the way $h$ is calculated (less efficient, more accurate).
\end{itemize}

\subsection{Version 1.2.0: November 12, 2014}

\noindent \verb|EstOptions.m|
\begin{itemize}
\item Fixed typo in warning message.
\item Changed the order in which \verb|R_psi| matrices are checked for redundancies/errors; \verb|restrictDB| with \verb|R_psi| non-empty had to be moved to before \verb|[1 -1]| is imposed.
\item Added a check to make sure that restrictions on $\psi$ in the model \verb|restrictDB| are imposed correctly.
\item Added option \verb|CalcSE| to turn off calculation of standard errors for faster computation.
\end{itemize}

\noindent \verb|LagSelect.m|
\begin{itemize}
\item Turned off calculation of standard errors.
\end{itemize}

\noindent \verb|RankTests.m|
\begin{itemize}
\item Turned off calculation of standard errors.
\item Fixed typo in output (un'r'estriced).
\end{itemize}

\noindent \verb|FCVARestn.m|
\begin{itemize}
\item Fixed typo in output (un'r'estriced).
\item Changed the way that optimization is performed when linear restrictions on $d,b$ are imposed. 
\item Transposed $(d,b)$ in case of full identification (\verb|R_psi| has two restrictions); this is done to match the way that the rank test results are stored.
\item Fixed the adjustment of \verb|UB| and \verb|LB| after grid search because it interfered with the $d \geq b$ constraint.
\item Removed the fast inversion of Hessian for calculating standard errors because it wasn't precise.
\item Changed the way that the commutation matrix enters in the translation from $\text{vec}(\alpha ^\prime)$ to $\text{vec}(\alpha)$ and vice versa in the rank condition for identification.
\end{itemize}

\noindent \verb|FCVARlike.m|
\begin{itemize}
\item Adjusted the way that likelihood is calculated when linear restrictions on $d,b$ are imposed. 
\item Changed how \verb|constrained| option is imposed, regardless of linear restrictions.
\end{itemize}

\noindent \verb|GetBounds.m|
\begin{itemize}
\item Added function for calculating upper and lower bounds.
\end{itemize}

\noindent \verb|GetParams.m|
\begin{itemize}
\item Changed matrix inversion method to more precise method (i.e., now use inv() instead of \ and / in low dimension situations).
\end{itemize}

\noindent \verb|LikeGrid.m|
\begin{itemize}
\item Added the variable \verb|phi| in the loop. Now, \verb|phi| goes into \verb|FCVARlike| and can be either a singleton or 2x1 vector. \verb|db| is reserved for \verb|FCVARlikeMU| which does not make the translation from \verb|phi| to \verb|db| automatically. It is also used in the output for the waitbar, fixing an issue where the waitbar displayed \verb|d=b=phi| in the case of a linear restriction.
\end{itemize}

\noindent \verb|RstrctOptm_Switch.m|
\begin{itemize}
\item Added new function that replaces \verb|RstrctOptm.m| and \verb|rLike.m|, which were called for estimation when either $\alpha$ or $\beta$ or both were restricted.
\item This function uses a switching algorithm from \cite{Boswijk2004} and performs much better than the previous algorithm which used unconstrained numerical optimization. 
\end{itemize}

\subsection{Version 1.2.1: November 17, 2014}

\noindent \verb|LikeGrid.m|
\begin{itemize}
\item Fixed a problem when the grid search finds a non-unique maximum of the likelihood. 
\end{itemize}

\noindent \verb|RankTests.m|
\begin{itemize}
\item Fixed a bug where the wrong value of $b$ was being used to calculate the $P$-value.
\item Changed the output from a matrix to a Matlab struct.
\end{itemize}

\noindent \verb|FCVARforecast.m|
\begin{itemize}
\item Changed \verb|rhoHatUNR| to \verb|xiHat| to use the same notation throughout.
\end{itemize}

\noindent \verb|FCVARestn.m|
\begin{itemize}
\item Adjusted output of roots of characteristic polynomial to line them up properly.
\end{itemize}

\subsection{Version 1.2.2: November 19, 2014}

\noindent \verb|FCVARestn.m|
\begin{itemize}
\item Fixed a problem with starting values from the grid search when $d,b$ are unrestricted and level parameters are included.
\end{itemize}

\subsection{Version 1.2.3: July 21, 2015}

\noindent \verb|HypoTest.m|
\begin{itemize}
\item Fixed a typo in the function description.
\end{itemize}

\noindent \verb|mv_wntest.m|
\begin{itemize}
\item Added specified lag to output.
\end{itemize}

\noindent \verb|Lag_select.m|
\begin{itemize}
\item Added model specification to output.
\end{itemize}

\noindent \verb|FreeParameters.m| (function nested in \verb|FCVARestn.m|)
\begin{itemize}
\item Set \verb|fDB = 2| instead of \verb|fDB = 1 + |$\sim$\verb|opt.restrictDB| because regardless of option \verb|restrictDB|, \verb|opt.R_psi| is updated to incorporate that restriction and this results in a double count. This bug probably did not affect previous hypothesis testing as long as tests were nested within a particular $(d,b)$ model. Also, if restrictions were imposed using \verb|R_psi| directly, instead of \verb|restrictDB|, then the program returned the correct number of free parameters. 
\end{itemize}

\subsection{Version 1.3.0: September 9, 2015}
\noindent \verb|FCVARestn.m|
\begin{itemize}
\item Moved all nested functions to \verb|Auxiliary| folder.
\end{itemize}

\noindent \verb|FCVARforecast.m|
\begin{itemize}
\item Deleted all nested functions (these are now in \verb|Auxiliary| folder).
\end{itemize}

\noindent \verb|replication_JNP2014.m|
\begin{itemize}
\item Added line that adds the path of \verb|Auxiliary| folder with all necessary functions.
\item Removed forecasting subsection.
\item Added variable to store output from rank tests.
\end{itemize}


\noindent Added the following new files/functions:

\noindent \verb|FCVARboot.m|
\begin{itemize}
\item Performs wild bootstrap for hypothesis tests on parameters.
\end{itemize}
\noindent \verb|FCVARbootRank.m|
\begin{itemize}
\item Performs wild bootstrap for rank tests.
\end{itemize}
\noindent \verb|FCVARsim.m|
\begin{itemize}
\item Generates samples with errors drawn from Normal distribution. 
\end{itemize}
\noindent \verb|FCVARsimBS.m| (Auxillary function)
\begin{itemize}
\item Generates (wild) bootstrap samples.
\end{itemize}
\noindent \verb|MoreExamples.m|
\begin{itemize}
\item Contains examples of forecasting, bootstrapping, simulation.
\end{itemize}


\subsection{Version 1.3.1: December 7, 2015}
\noindent \verb|LagSelect.m|
\begin{itemize}
\item Fixed calculation of BIC to use (T - opt.N) for the penalty.
\end{itemize}
\noindent \verb|CharPolyRoots.m|
\begin{itemize}
\item Imposed axis equal in plot.
\end{itemize}


\subsection{Version 1.3.2: December 11, 2015}
\noindent \verb|FCVARestn.m|
\begin{itemize}
\item Fixed a bug in the calculation of standard errors when $(d,b)$ are both unrestricted but restrictions are imposed on other parameters.
\end{itemize}


\subsection{Version 1.3.3: January 26, 2016}
\noindent \verb|LagSelect.m|
\begin{itemize}
\item Added missing index for multivariate white noise test variable.
\end{itemize}


\subsection{Version 1.4.0: April 15, 2016}

\noindent New features: Added \verb|opt.LocalMax| and \verb|opt.LineSearch|, see Sections \ref{subsec choosing options} and \ref{restricted models}, respectively. Also added the ability to set separate bounds for the parameter space for $d$ and $b$, see Section \ref{subsec choosing options}.

\noindent \verb|EstOptions.m|
\begin{itemize}
\item Update assignment of upper and lower bounds for $d$ and $b$ to allow for different values for the two parameters. Specifying just one upper and lower bound is still permitted for backwards compatibility.
\item Added option \verb|opt.LocalMax|.
\item Added additional checks for validity of restrictions on $d$ and/or $b$.
\end{itemize}
\noindent \verb|FCVARestn.m|
\begin{itemize}
\item Changed the names of H matrices because there are multiple values for the same variable, left notation H for Hessian.
\item Changed output to report upper and lower bounds for both $d$ and $b$ regardless of the restrictions.
\item Added notification about restrictions to output. 
\end{itemize}
\noindent \verb|GetBounds.m|
\begin{itemize}
\item Changed setting of bounds to allow for different limits on parameter space for $d$ and $b$.
\end{itemize}
\noindent \verb|GetParams.m|
\begin{itemize}
\item Changed input sent to switching algorithm. Now $\alpha$ and $\beta$ are not normalized.
\end{itemize}
\noindent \verb|LagSelect.m|
\begin{itemize}
\item Added asterisks to output next to values that minimize the AIC and BIC.
\end{itemize}
\noindent \verb|LikeGrid.m|
\begin{itemize}
\item Modified code to allow for different upper and lower limits on parameter space for $d$ and $b$.
\item Created separate grids for $d$ and $b$ and changed the way that they are indexed as well as how the total number of iterations is counted based on \verb|opt.constrained| (i.e., on whether $d \geq b$ is imposed).
\item Added code to accommodate the new option \verb|opt.LocalMax|.
\end{itemize}
\noindent \verb|RstrctOpt_Switch.m|
\begin{itemize}
\item Incorporated a line search for (much) faster and more accurate convergence of switching algorithm when restrictions are imposed on either $\alpha$ or $\beta$.
\end{itemize}


\subsection{Version 1.4.0a: May 1, 2018}

\noindent Added some new references and updated some existing ones. No changes to the code.


\cleardoublepage
\phantomsection
\addcontentsline{toc}{section}{References}
\bibliography{references}
\bibliographystyle{chicago}


\end{document}
